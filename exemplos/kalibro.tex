\section{Kalibro Metrics}
\label{sec:kalibro}

O Kalibro Metrics é um serviço web livre, projetado para se conectar a qualquer
ferramenta de coleta de métricas de código-fonte. Seu objetivo é
fornecer uma avaliação de fácil entendimento da qualidade (``saúde'') do
software analisado.
%

O Kalibro Metrics permite que um especialista em métricas (ou qualquer usuário)
especifique conjuntos de intervalos de valores para cada métrica fornecida pela
ferramenta base, além de possibilitar a criação de novas métricas.
%
Com essa abordagem, pretendemos difundir o uso de métricas de código-fonte, pois
com um conjunto de intervalos (ou seja, os valores frequentes) e suas
respectivas interpretações, qualquer desenvolvedor pode explorá-las e melhor
entendê-las.

Argumentamos que é desejável um software que colete métricas e associe seus
resultados a alguma interpretação, tornando-as mais fáceis de entender.
%
Ao constatar dificuldades nas ferramentas de métricas, estabelecemos
alguns requisitos para um software que vise a difundir o uso de métricas
de código-fonte:

\begin{enumerate}

\item \textbf{Intervalos de aceitação:} Deve permitir o uso de múltiplos intervalos
para fornecer diferentes interpretações sobre os valores das métricas.
%
Por exemplo, usando o Eclipse Metrics \citep{Metrics2005} é possível
configurar valores mínimo e máximo (um intervalo de aceitação) para cada métrica
que ele fornece, e associar uma dica de correção quando o resultado para um
método, classe ou pacote está fora desse intervalo.
%
Os intervalos devem ser configuráveis, pois as métricas em geral não possuem
valores de referência absolutos ou universalmente aceitos, como discutimos nesta
tese.
%
Os valores ideais podem variar de acordo com vários fatores, como a linguagem e
o domínio de aplicação.
 
  \item \textbf{Extensível:} Não deve se limitar a apenas uma linguagem de
programação, nem a um conjunto pré-definido de métricas.
%
Deve fornecer interface clara para adicionar suporte a novas linguagens de
programação, pois isso pode atrair uma gama maior de usuários em potencial para
sugerir intervalos de acordo com suas experiências em linguagens específicas.
%
Além disso, o usuário deve ser capaz de configurar o conjunto de métricas que
deseja analisar.
%
Muitos coletores de métricas calculam e exibem o resultado de todas as métricas
de seu repertório, o que consome  mais recursos que o necessário (i.e., tempo e
memória) e polui a exibição.
%
Deve haver suporte para inclusão de novas métricas, a partir de diferentes
coletores de métricas ou a partir das métricas existentes.
%
O índice de manutenibilidade \citep{VanDoren1997} é um exemplo de métrica criada
como combinação de métricas básicas.
%
Outro exemplo é a complexidade estrutural, explicada na
Seção~\ref{sec:metricas-selecionadas}, que é o produto de uma métrica de
coesão com uma de acoplamento.

\item \textbf{Comparação entre projetos:} A ferramenta deve dar suporte a
comparação de características específicas do código-fonte de vários projetos
através de métricas, dando suporte aos pesquisadores e ajudando nas decisões de
adoção de um determinado software.

\item \textbf{Software livre mantido ativamente:} A ferramenta deve ser
software livre, disponível sem restrições e mantida ativamente.
%
Assim, qualquer um pode contribuir com seu desenvolvimento e modificá-la de
acordo com suas necessidades.
%
Isso também permite que pesquisadores repliquem completamente estudos e
resultados -- algo que está na concepção das publicações científicas: serem
reprodutíveis.

\end{enumerate}

\begin{table}[hbt]
\begin{center}
\begin{tabular}{|l|c|c|c|c|c|}
  \hline
  Ferramenta                 & Linguagens          & \textbf{Intervalos} & \textbf{Extensível} & \textbf{Comparação} & \textbf{Livre} \\
  \hline\hline

  \textbf{Analizo}           & C, C++, Java        & Não                 & Sim             & Não             & Sim \\
  \hline

  \textbf{Analyst4j}         & Java                & Sim                 & Não             & Não             & Não \\
  \hline

  \textbf{CCCC}              & C++, Java           & Não                 & Não             & Não             & Sim \\
  \hline

  \textbf{Checkstyle}        & Java                & Sim                 & Não             & Não             & Sim \\
  \hline

  \textbf{CK Java Metrics}   & Java                & Não                 & Não             & Não             & Sim \\
  \hline

  \textbf{CMetrics}          & C                   & Não                 & Sim             & Não             & Sim \\
  \hline

  \textbf{Cscope}            & C                   & Não                 & Não             & Não             & Sim \\
  \hline

  \textbf{CVSAnaly}            & C, C++, Python      & Não                 & Sim             & Não             & Sim \\
  \hline

  \textbf{Dependency Finder} & Java                & Não                 & Não             & Não             & Sim \\
  \hline

  \textbf{Infusion}          & C, C++, Java        & Sim                 & Não             & Sim             & Não \\
  \hline


  \textbf{JaBUTi}            & Java                & Não                 & Não             & Não             & Sim \\
  \hline

  \textbf{MacXim}            & Java                & Não                 & Não             & Não             & Sim \\
  \hline

  \textbf{Metrics}           & Java                & Sim                 & Não             & Não             & Sim \\
  \hline

  \textbf{OOMeter}           & Java,C\#            & Não                 & Não             & Não             & Não \\
  \hline

\textbf{Understand for Java} & Java                & Não                 & Não             & Não             & Não \\
  \hline

\textbf{VizzAnalyzer}        & Java                & Não                 & Não             & Não             & Não \\
  \hline

\end{tabular}
\caption{Ferramentas existentes versus requisitos definidos}
\label{tab:ferramentas}
\end{center}
\end{table}


Ao longo deste trabalho, conhecemos, estudamos e/ou utilizamos 16
ferramentas\footnote{Ferramentas estudadas:
\url{analizo.org},
\url{codeswat.com},
\url{cccc.sourceforge.net},
\url{checkstyle.sourceforge.net},
\url{http://metricsgrimoire.github.io/CVSAnalY/},
\url{spinellis.gr/sw/ckjm},
\url{tools.libresoft.es/cmetrics},
\url{cscope.sourceforge.net},
\url{depfind.sourceforge.net},
\url{intooitus.com/products/infusion},
\url{ccsl.icmc.usp.br/pt-br/projects/jabuti},
\url{qualipso.dscpi.uninsubria.it/macxim},
\url{metrics.sourceforge.net},
\url{ccse.kfupm.edu.sa/~oometer/oometer},
\url{scitools.com},
\url{arisa.se}.
}para análise de código-fonte.
%
Na Tabela \ref{tab:ferramentas}, as ferramentas estudadas são comparadas com
os requisitos. Percebe-se que nenhuma delas preenche todos os requisitos.
%
Dessa forma, defendemos que para promover o uso de métricas de código-fonte por
desenvolvedores em geral, é necessária uma ferramenta que sistematize a
avaliação do código-fonte.
%
Adicionalmente, com exceção do software restrito Infusion, que conhecemos quando
já estávamos em um estágio avançado de desenvolvimento de nossa pesquisa, os coletores
de métricas de código-fonte que analisamos possuem, em geral, uma ou mais
dessas carências:

\begin{enumerate}

\item \textbf{Associação entre resultados numéricos e forma de interpretá-los:}
%
Ferramentas de métricas frequentemente mostram seus resultados como valores
numéricos isolados para cada métrica.

\item \textbf{Flexibilidade nessa interpretação:}
%
Algumas ferramentas mostram indicadores binários (bom ou ruim) para os valores
das métricas, mas não há forma de configurar os valores de referência.

\item \textbf{Flexibilidade para funcionar com diferentes linguagens:}
%
As ferramentas que não pecam nos quesitos anteriores foram desenhadas
especificamente para uma só linguagem de programação.

\item \textbf{Criação de novas métricas}: capacidade de estender seu uso com
novas métricas, derivadas das existentes na ferramenta.

\end{enumerate}

Baseados nesses conceitos desenvolvemos o que é hoje o Kalibro Metrics:
um serviço responsável pelo suporte à interpretação das métricas.
%
Do ponto de vista de implementação, foi definida uma interface de conexão
simples para delegar a coleta de valores para ferramentas de análise de
código-fonte existentes.
%
Dessa forma, o Kalibro pode se conectar às ferramentas mais interessantes
para o contexto de cada software analisado.

Como primeira ferramenta base, foi escolhida a Analizo, pelas características
comentadas na Seção \ref{sec:analizo}.
%
Posteriormente, o CheckStyle foi integrado para termos uma ferramenta
especialista na análise de código escrito em Java no projeto Mezuro, fornecendo
métricas mais específica do paradigma de orientação a objetos para código
Java.

O Kalibro foi desenvolvido de forma modular e desacoplada, para facilitar
modificações, incrementos e incorporações de outras ferramentas e diferentes
interfaces de usuário (camadas de visualização).
%
Toda a lógica, processamento e persistência foram isolados em um núcleo
independente de interface com o usuário, que chamamos de \textit {KalibroCore},
e o acesso remoto está implementado na interface do serviço,
\textit{KalibroService}.
%
Os detalhes da arquitetura da Kalibro Metrics e uma discussão das funcionalidades
do projeto Mezuro do ponto de vista da arquitetura do Kalibro estão no
Apêndice \ref{apend:kalibro}.

Por fim, depois do primeiro protótipo do Mezuro, o desenvolvimento do
Kalibro Metrics foi associado ao desenvolvimento do plugin Mezuro.
%
Na prática, o plugin Mezuro funciona como uma camada de visualização dos resultados e
funcionalidades do Kalibro Metrics.

