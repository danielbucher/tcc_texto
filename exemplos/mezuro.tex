\section{Mezuro}
\label{sec:mezuro}

Desde a proposta inicial deste doutorado, pensamos em uma ferramenta com a capacidade
de reunir periodicamente os dados de projetos de software livre, visitando seu
código-fonte diretamente a partir de seus respectivos repositórios, obtendo vários tipos de
dados e gerando avaliações sobre a qualidade do código de forma semi-automática.
%
O objetivo era o de tornar essas avaliações públicas de forma a ajudar empresas,
governos e indivíduos a decidir sobre a utilização ou não de um software livre
e a auxiliar na escolha de uma das soluções dentre as várias possíveis, do ponto vista
técnico, pensando em sua manutenabilidade, além de
%
promover a melhoria da qualidade do software livre
produzido por desenvolvedores e fomentar a confiança nos produtos.

A ideia central dessa ferramenta passa pelo conceito de configuração de métricas.
Uma configuração é um conjunto de métricas em que cada métrica está associada a um grupo
de intervalos de valores possíveis para essa métrica e às interpretações sobre o significado
desses intervalos.
%
Essa configuração é criada e personalizada pelo usuário (um desenvolvedor ou equipe
que irá monitorar um determinado código).
%
Pensamos primeiramente um intervalo como algo que, além de possuir um valor de
início e fim, tem uma avaliação (na forma de uma nota) associada, uma cor para exibição e
comentários pertinentes a exibir quando o valor da métrica estiver dentro dele.
%
Isso possibilita associar uma avaliação qualitativa ao valor obtido
como resultado do cálculo da métrica.

Uma configuração de métricas específica é associada a um dado projeto de software.
%
Quando seu código é analisado, apenas os resultados das métricas
contidas na configuração associada a ele são calculados, com seus valores associados
aos intervalos correspondentes.
%
Assim, o usuário pode criar várias configurações e utilizá-las para atender às suas necessidades
específicas quanto ao código do projeto.

Cada métrica de uma configuração possui um peso. Com a nota dos intervalos e o
peso de cada métrica, se calcula uma nota geral do projeto, pacote, classe ou
método, com uma média ponderada.
%
Desse modo, além de oferecer uma forma de interpretar resultados de métricas, as
configurações também servem como critério específico de avaliação e comparação
entre projetos de software.
%

Em suma, a possibilidade de definir configurações de métricas é a parte mais importante da flexibilidade
da plataforma que propusemos e desenvolvemos.
%
A configuração utilizada por um líder de projeto para acompanhar a qualidade
do código pode sofrer melhorias ao longo do tempo e, nesse sentido,
esse ambiente de monitoramento semi-automático proposto também foi pensado para
compartilhar essas configurações, facilitando uma discussão em torno delas em
rede e sua subsequente evolução.

Apostando nessas ideias é que investimos boa parte dos esforço deste doutorado
no desenvolvimento e na coordenação de um projeto de uma rede sócio-técnica que
tem o objetivo de ser um ambiente aberto e colaborativo de avaliações de
código-fonte e aprendizado do ``estado-da-prática'' dos projetos de software
livre.
%
Essa rede é o Mezuro, construído sobre a plataforma de redes sociais
Noosfero\footnote{\url{noosfero.org}}, através de um plugin que se comunica com
o Kalibro Metrics Service.
%
O Mezuro é norteado pela extração automatizada de métricas de
código-fonte e por uma maneira objetiva de interpretar os seus valores, para que
os engenheiros de software possam monitorar as características específicas de
seu código.

Antes de partirmos para o desenvolvimento do Mezuro, estudamos algumas
plataformas correlatas\footnote{Projetos correlatos:
  \url{flossmetrics.org},
  \url{ohloh.net},
  \url{qualoss.org},
  \url{sqo-oss.org},
  \url{qsos.org},
  \url{fossology.org},
  \url{code.google.com/p/hackystat}.
}:

\begin{itemize}

\item \textbf{FLOSSMetrics} (\textbf{Free/Libre Open Source Software Metrics})
é um projeto que utiliza metodologias e ferramentas existentes para fornecer um
grande banco de dados com informações sobre o desenvolvimento de software livre.

\item \textbf{Ohloh} é uma plataforma que oferece um conjunto de serviços web
e um sistema web para comunidade de software livre, que visa prover uma
visão geral da evolução dos projetos de software livre em desenvolvimento.

\item \textbf{Qualoss} (\textbf{Quality in Open Source Software}) é uma
metodologia para automatizar a medição da qualidade de projetos de software
livre, usando ferramentas para analisar o código-fonte e as informações dos 
repositórios dos projetos.

\item \textbf{SQO-OSS} (\textbf{Software Quality Assessment of Open Source
Software}) fornece um conjunto de ferramentas para análise e avaliação
comparativa de projetos de software livre.

\item \textbf{QSOS} (Qualification and Selection of Open Source Software)
é uma metodologia baseada em quatro etapas: definição de referência utilizada;
avaliação de software; qualificação dos usuários em contexto específico;
seleção e comparação de software.

\item \textbf{FOSSology} (Advancing open source analysis and development)
é um projeto que fornece um banco de dados gratuito com informações sobre
licenças de software livre.

\item \textbf{HackyStat} é um ambiente para visualização, análise e
interpretação do processo de desenvolvimento de software e dados do produto
de software.

\end{itemize}

Como já mencionado, nossas ferramentas também foram desenvolvidas no contexto do
projeto QualiPSo (\textit{Trust and Quality in Open Source
Systems})~\footnote{Projeto Qualipso: \url{qualipso.org}},
que nos ajudou com os estudos para estimular o uso do software livre
e adoção das suas práticas de desenvolvimento dentro da indústria de software.
%
Com nossa experiência de pesquisa no projeto QualiPSo e a análise dos
outros projetos relacionados citados acima, desenvolvemos a plataforma Mezuro.

Nenhuma das plataformas citadas, quando as avaliamos na concepção do projeto
Mezuro, já com o desenvolvimento do primeiro protótipo da parte de configuração
de métricas na ferramenta Crab~\citep{meirelles:sbes09}, tem o objetivo de
ficar monitorando constantemente a evolução das métricas de código-fonte em si
dos projetos de software livre.
%
Como apresentamos, o Mezuro colocar, também, nas mãos dos usuários
o poder de definir quais métricas quer monitorar, quando quer monitorar e até
quais as interpretações ele quer se basear para sua avaliação sobre as métricas.
%
De qualquer forma, o Mezuro não propõe em ser um ``concorrente'', mas sim
complementar as abordagens das demais plataformas.
%
Por exemplo, com a experiência da nossa interação com o Olhoh, que apenas tem
apenas a métrica de número de linhas de código como informação da análise
estática do código-fonte, avaliamos ser factível uma integração com essa plataforma, uma vez que podemos integrar
a comunicação da API deles com o nosso \textit{web service} para obtermos
dados ``sociais'' das comunidade de software livre; bem como, eles podem
usar o Kalibro Metrics (nossa instância ou uma própria deles, uma vez que todas
as nossas ferramentas são livres), passando a URL do repositório do projetos em
análise, para obter os valores das métricas de código-fonte, e assim relatar
mais informações técnicas à respeito dos projetos avaliados no Ohloh.

%-------------------------------------------------------------------------------

\subsection{Funcionalidade}

Uma analogia útil para se ententer a utilidade do Mezuro é lembrar que as
pessoas melhoram suas habilidades de escrita à medida em que leem bons livros.
%
Da mesma forma, os engenheiros de software podem aprender e melhorar seus códigos
ao encontrar projetos relacionados que possuem bons códigos (ou seja, lendo bons
códigos), comparando suas implementações e características.
%
Para que isso seja possível do ponto de vista do monitoramento do código-fonte
de projetos de software livre, as funcionalidades implementadas até aqui no
projeto Mezuro são:

\begin{itemize}

\item Download de código-fonte a partir de repositórios do tipo
Subversion, Git, Mercurial, Baazar e CVS.
%
\item Criação de configurações, ou seja, conjuntos pré-definidos de
métricas relacionadas para serem usadas na avaliação do código-fonte dos
projetos.
%
\item Criação de intervalos qualitativos associados aos valores das métricas.
%
\item Criação de novas métricas (via JavaScript) baseadas em outras
fornecidas pelas ferramentas (coletores de métricas) integradas à plataforma.
%
\item Cálculo dos resultados estatísticos para partes do software com
granularidade alta (e.g. a média do número de linhas de código das classes
por pacote).
%
\item Cálculo de uma nota para o código-fonte analisado, baseado em pesos
atribuídos para as métricas e seus intervalos associados, o que permite a
comparação entre projetos analisados sob a mesma configuração.
%
\item Interpretação amigável e textual dos valores das métricas com a
definição de grupos de leitura e interpretação (\textit{reading groups}) com
cores e comentários para cada intervalo definido.

\item Visualização do histórico de cada métrica nos projetos avaliados.

\end{itemize}

\begin{figure}[!h]
\centering
\includegraphics[scale=.45]{figures/chapter5/mezuro1.png}
\caption{Painel de controle de uma comunidade na rede Mezuro.}
\label{mezuro1}
\end{figure}

\begin{figure}[!h]
\centering
\includegraphics[scale=.45]{figures/chapter5/mezuro2.png}
\caption{Cadastrando um novo projeto no Mezuro.}
\label{mezuro2}
\end{figure}

O Mezuro tem dois grupos de funcionalidades: em nível de projeto e em nível de configuração
(que também contém os grupos de leitura e interpretação dos intervalos para as
métricas).
%
O projeto, ou seja, um tipo de conteúdo na rede que refere-se a um software que
terá seu código-fonte monitorado, pertence a um tipo de perfil da rede chamado
de ``comunidade''.
%
Dessa forma, ao se cadastrar na rede Mezuro para monitorar as métricas de
código-fonte de um software livre, o usuário deve criar uma comunidade
para o projeto que deseja monitorar (ou ainda, procurar uma comunidade
correspondente ao software livre de interesse).
%
No painel de controle de uma comunidade está disponível o tipo de conteúdo
chamado ``Mezuro Project'', como ilustrado na Figura~\ref{mezuro1}.

\begin{figure}[!h]
\centering
\includegraphics[scale=.45]{figures/chapter5/mezuro3.png}
\caption{Adicionando um repositório ao projeto no Mezuro.}
\label{mezuro3}
\end{figure}

\begin{figure}[!h]
\centering
\includegraphics[scale=.45]{figures/chapter5/mezuro4.png}
\caption{Cadastrando as informações do repositório no projeto do Mezuro.}
\label{mezuro4}
\end{figure}


Para criar um Projeto Mezuro, basta ao usuário informar um nome e uma descrição para o software,
como exemplificado na Figura~\ref{mezuro2}.
%
Posteriormente, deverá adicionar pelo menos um repositório de código-fonte
correspondente ao projeto, como mostrado na sequência das Figuras~\ref{mezuro3}
e~\ref{mezuro4}. Vale observar que não é excepcional que projetos de software
livre tenham mais de um repositório, muitas vezes correspondentes aos
seus subprojetos.


\begin{figure}[!h]
\centering
\includegraphics[scale=.45]{figures/chapter5/mezuro5.png}
\caption{Visualizando a coleta das métricas no Mezuro.}
\label{mezuro5}
\end{figure}

\begin{figure}[!h]
\centering
\includegraphics[scale=.45]{figures/chapter5/mezuro6.png}
\caption{Visualizando os dados do processo de coleta das métricas}
\label{mezuro6}
\end{figure}

\begin{figure}[!h]
\centering
\includegraphics[scale=.45]{figures/chapter5/mezuro7.png}
\caption{Visualizando os resultados de cada uma das métricas.}
\label{mezuro7}
\end{figure}

São cadastrados também (i) a licença do código disponibilizado nos repositórios
informados, (ii) a periodicidade (em dias) em que as métricas serão coletadas,
(iii) o tipo de repositório, (iv) a URL do repositório em questão e
(v) a configuração de avaliação das métricas. Essas informações
são salvas na rede e enviadas ao Kalibro Metrics.
%
O Kalibro posiciona a coleta das métricas dos repositórios cadastrados em uma fila
para ser processados; esse processamento pode levar horas dependendo da
quantidade classes e módulos do software no repositório.
%
Após a coleta das métricas e processamento de acordo com a configuração
selecionada, os resultados podem ser visualizados como exemplicado na sequência
das Figuras~\ref{mezuro5},~\ref{mezuro6} e~\ref{mezuro7}. 

\begin{figure}[!h]
\centering
\includegraphics[scale=.45]{figures/chapter5/mezuro7-1.png}
\caption{Visualizando o gráfico com o histórico das métricas.}
\label{mezuro7.1}
\end{figure}

Com as respostas enviadas do Kalibro para o Mezuro plugin, é possível navegar
entre os diretórios (pacotes) e arquivos (classes e módulos) do repositório
analisado para observar individualmente os valores das métricas para cada
parte do software em questão.
%
Também, para cada métrica é possível observar a evolução dos seu valores
ao longo do tempo (conforme a quantidade de vezes que um projeto foi analisado,
de acordo com a periodicidade cadastrada) através de gráficos, como
exemplificado na Figura~\ref{mezuro7.1}.
%
Destacamos que aspectos de visualização de software não fazem parte do escopo desta tese de
doutorado, mas ao longo do desenvolvimento do projeto Mezuro identificamos
a necessidade de aplicá-los. Assim, o uso de técnicas e abordagens de
visualização de software estão sendo estudados para serem aplicados na
continuidade deste projeto.


\begin{figure}[!h]
\centering
\includegraphics[scale=.45]{figures/chapter5/mezuro8.png}
\caption{Painel de controle de um usuário na rede Mezuro.}
\label{mezuro8}
\end{figure}

O outro tipo de conteúdo do Mezuro Plugin é o ``Mezuro Configuration'' (que tem
agregado o ``Mezuro Reading Group'').
%
Diferentemente de um ``Projeto Mezuro'', uma ``Configuração do Mezuro'' foi pensada
para ser utilizada por especialistas em métricas de código-fonte que desejam
cadastrar valores de referência ou frequentes para as métricas, de acordo com
suas experiências e necessidades.
%
Nesse ponto é que está a utilização prática e automatizada das análises dos valores
frequentes dos projetos discutidos no Capítulo \ref{cap:estudos}.
%
Os valores frequentes de cada métrica em cada projeto, dentro das faixas de
percentis encontrados, serão cadastrados no Mezuro para disponibilizarmos
um conjunto inicial de configurações de métricas na rede de monitoramento
de código-fonte.

Esse tipo de conteúdo é disponibilizado no perfil pessoal
de cada usuário da rede Mezuro, como ilustrado na Figura~\ref{mezuro8}.
%
Deseja-se que as configurações sejam associadas às pessoas para que os demais
usuários conheçam quem definiu cada configuração.
%
Certamente, as configurações disponibilizadas por pessoas reconhecidas dentro
de uma comunidade de software livre tenderão a ser mais usadas no monitoramento
das métricas dos projetos associados a essa comunidade.


\begin{figure}[!h]
\centering
\includegraphics[scale=.45]{figures/chapter5/mezuro9.png}
\caption{Criando uma nova configuração do Mezuro.}
\label{mezuro9}
\end{figure}

Para cadastrarmos uma nova Configuração do Mezuro, informamos um nome e uma
descrição para identificá-la na rede, como exemplificado
na Figura~\ref{mezuro9}.
%
O usuário ainda tem a opção de clonar uma configuração já existente na
rede para fazer modificações pontuais, mas sem a necessidade de criar uma
configuração do zero. 

\begin{figure}[!h]
\centering
\includegraphics[scale=.45]{figures/chapter5/mezuro10.png}
\caption{Adicionando métricas à configuração do Mezuro.}
\label{mezuro10}
\end{figure}


\begin{figure}[!h]
\centering
\includegraphics[scale=.45]{figures/chapter5/mezuro11.png}
\caption{Selecionando uma ferramenta base integrada ao Mezuro.}
\label{mezuro11}
\end{figure}


\begin{figure}[!h]
\centering
\includegraphics[scale=.45]{figures/chapter5/mezuro12.png}
\caption{Selecionando métricas da Analizo numa configuração do Mezuro.}
\label{mezuro12}
\end{figure}


\begin{figure}[!h]
\centering
\includegraphics[scale=.45]{figures/chapter5/mezuro13.png}
\caption{Selecionando métricas do Checkstyle numa configuração do Mezuro.}
\label{mezuro13}
\end{figure}

Os próximos passos, como mostrado na sequência das Figuras~\ref{mezuro10},
~\ref{mezuro11},~\ref{mezuro12} e~\ref{mezuro13}, consistem em, dentre as
ferramentas base, no caso Analizo e CheckStyle, selecionar uma métrica e
configurá-la (e assim por diante para cada métrica que o usuário desejar configurar).
%
O Kalibro envia para o Mezuro plugin a lista de métricas que são calculadas
em cada ferramenta base. Ao selecionar uma das métricas, o usuário deve
definir a configuração da métrica selecionada.

\begin{figure}[!h]
\centering
\includegraphics[scale=.45]{figures/chapter5/mezuro15.png}
\caption{Configurando uma métrica.}
\label{mezuro15}
\end{figure}

\begin{figure}[!h]
\centering
\includegraphics[scale=.45]{figures/chapter5/mezuro16.png}
\caption{Definindo um intervalo para uma métrica.}
\label{mezuro16}
\end{figure}


\begin{figure}[!h]
\centering
\includegraphics[scale=.45]{figures/chapter5/mezuro18.png}
\caption{Métrica configurada com um intervalo cadastrado.}
\label{mezuro18}
\end{figure}

Para cada métrica que deseja adicionar em sua configuração do Mezuro, 
o usuário especialista em métricas deve informar, como
exemplificado na Figura~\ref{mezuro15},
%
(i) um código para a métrica, que é usado como identificador dessa métrica
caso ela seja utilizada no cálculo de uma métrica composta;
%
(ii) uma forma de agregação (média, máximo, mínimo etc.) quando tal
métrica for contabilizada em uma parte de um software com maior granularidade;
%
(iii) um grupo de leitura e interpretação de intervalos da métricas
(vamos explicar mais adiante essa funcionalidade).
%
Após salvar essas informações, o Mezuro permite que intervalos de valores
e a interpretação para cada um desses intervalos sejam adicionados à
configuração da métrica em questão, como ilustrado na sequência das
Figuras~\ref{mezuro16} e~\ref{mezuro18}.
%
Para cada intervalo, o usuário deve escolher um rótulo, que vem pré-definido
de acordo com o grupo de leitura e interpretação selecionado. Posteriormente,
ele deve atribuir um valor inicial e um final para o intervalo.
%
Por fim, deve inserir um comentário referente à interpretação e possíveis ações
que podem ser sugeridas quando o valor da métrica estiver no referido intervalo.

\begin{figure}[!h]
\centering
\includegraphics[scale=.45]{figures/chapter5/mezuro21.png}
\caption{Configurando um grupo de leitura e interpretação de intervalos.}
\label{mezuro21}
\end{figure}

\begin{figure}[!h]
\centering
\includegraphics[scale=.45]{figures/chapter5/mezuro23.png}
\caption{Definindo um tipo de leitura.}
\label{mezuro23}
\end{figure}


Para entendermos melhor a configuração das métricas, podemos fazer uma comparação
com um paciente que faz um exame de sangue: ele precisa levar os resultados ao
seu médico, que vai usar seus conhecimentos para fazer uma leitura, indicando
quais índices estão saudáveis e quais são preocupantes, fazendo recomendações
ou prescrevendo algum tratamento.
%
Da mesma forma, resultados de métricas não são fáceis de interpretar por um
não especialista sem que sejam associados a uma interpretação que lhes dê
significado.
%
No Mezuro, essa interpretação é feita através de intervalos e de leituras desses
intervalos.

Como ilustrado na sequência das Figuras~\ref{mezuro21} e~\ref{mezuro23},
podemos cadastrar tais leituras para facilitar o seu reúso:
%
(i) O rótulo é uma forma verbal concisa de chamar atenção para o conteúdo de uma
interpretação. Exemplos: ``Saudável'', ``Ruim'', ``Complexo'', ``Belo'';
%
(ii) A cor chama atenção de forma visual, ajudando a identificar rapidamente
um resultado que esteja destoando dos outros, quando mostrados em conjunto.
%
A nota é um número, tornando possível comparar resultados e medir
a diferença entre esses resultados e seus valores esperados.
%
As leituras se agrupam naturalmente. Os grupos de leitura e interpretação
(\textit{Reading Groups}) dão suporte a esses agrupamentos dando a eles um nome
para facilitar o reúso.

Por fim, o projeto Mezuro foi pensado para o monitoramento de métricas de código-fonte
do ponto de vista de sua análise estática.
%
Entretanto, a flexibilidade provida pelo Kalibro Metrics permite que ele também
trate outros tipos de métricas, como as de cobertura de testes, fornecidas pela
ferramenta JaBUTi, e métricas de atividades do repositório, como o número de
contribuidores e \textit{commits}, coletadas pela CVSAnaly -- ambas ferramentas
estão sendo integradas ao Mezuro.
%
Da mesma forma, a integração com outros coletores permitirão o monitoramento de
projetos em outras linguagens de programação além de C, C++ e Java.
%
Por exemplo, com o CVSAnaly será possível o monitoramente de métricas de código-fonte
escrito em Python.

%------------------------------------------------------------------------------%

\section{Continuidade do Projeto Mezuro}


O projeto Mezuro e suas ferramentas deixaram de ser apenas uma contribuição
para esta tese de doutorado.
%
Além de ter colaborado diretamente com o projeto Qualipso e suas pesquisas
na área de qualidade de produto, atualmente o Mezuro é um projeto em si
do Núcleo de apoio às pesquisas em software livre da USP (NAPSoL-PRP-USP).

Antes disso, nossas ferramentas fizeram parte e colaboraram com outros trabalhos
do IME-USP.
%
Em 2009, colaborações na Kalibro fizeram parte do trabalho de 
fim de curso do aluno Carlos Morais.
%
A Analizo foi usada como estudo de caso do trabalho de conclusão de 
curso sobre um mapeamento entre os conceitos de código limpo e métricas de
código-fonte dos alunos João Machini e Lucianna Almeida, em 2010.
%
Na dissertação de mestrado de Hugo Corbucci, defendida em 2011 e que realizou
um estudo da relação entre métodos ágeis e software livre, a Analizo foi
utilizada para coletar algumas métricas também discutidas nesse estudo.


Além disso, o desenvolvimento do Kalibro Metrics e todos os conceitos que foram propostos
e implementados nele tornaram-se a base do trabalho de mestrado no IME-USP do mestrando
 Carlos Morais, previsto para 2013.
%
As questões de escalabilidade e desempenho do Kalibro Metrics, bem
como do projeto Mezuro, estão também sendo tratadas como objeto de estudo do
trabalho de conclusão de curso do graduando Diego Araújo, em 2013.
%
Ainda, a forma como o Mezuro ``orquestra'' os acessos aos \textit{end-points} 
do Kalibro Metrics e outras questões gerais de escalabilidade e desempenho
de \textit{web services} deverão ser um estudo de caso real para
a tese de doutorado do IME-USP na área de orquestração e coreografias de 
\textit{web services} do doutorando Paulo Moura (previsto para 2015).

Por fim, as ideias e a experiência no projeto Mezuro estão proporcionando uma
interação com o grupo do professor Michelle Lanza, da Universidade de Lugano,
através do aluno João Marco Silva, que está como aluno visitante na Suíça,
desenvolvendo trabalhos na área de visualização de software para aplicarmos
ao projeto Mezuro. No mesmo espírito, os resultados científicos desta tese de doutorado
serão apresentados ao grupo do professor Lanza para complementarmos nossas
abordagens sobre as análises de métricas de código-fonte.
%
Também estabelecemos, durante este doutorado, uma boa relação com
a Universidade Rey Juan Carlos, o que irá proporcionar um estágio ao
aluno Diego Araújo, na Espanha, com o professor Gregório Robles, para integramos
nossos projetos do ponto de vista prático.

