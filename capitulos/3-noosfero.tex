\chapter{Noosfero}


\section{Software Livre}

Software expressa uma solução abstrata dos problemas computacionais.
%
O software, em um sistema computacional, é o componente que contém o
conhecimento relacionado aos problemas a que a computação se aplica.
%
Por isso, o software é algo de interesse geral, uma vez que vários aspectos
relacionados a ele ultrapassam as questões técnicas, como por exemplo:
\begin{itemize}
\item O processo de desenvolvimento do software; 
\item Os mecanismos econômicos (gerenciais, competitivos, sociais, cognitivos etc.)
que regem esse desenvolvimento e seu uso;
\item O relacionamento entre desenvolvedores, fornecedores e usuários do
        software;
\item Os aspectos éticos e legais relacionados ao software.
\end{itemize}

O que define e diferencia o software livre do que podemos denominar de
software restrito passa pelo entendimento desses quatro pontos dentro do que é
conhecido como o \emph{ecossistema do software livre}.
%
O princípio básico desse ecossistema é promover a liberdade do usuário,
sem discriminar quem tem permissão para usar um software e seus limites de uso,
baseado na colaboração e num processo de desenvolvimento aberto.
%
Software livre é aquele que permite aos usuários usá-lo, estudá-lo, modificá-lo e
redistribui-lo, em geral, sem restrições para tal e prevenindo que não sejam
impostas restrições aos futuros usuários.
%

Normalmente, esse software existe por meio de projetos de desenvolvimento
que estão centradas em torno de algum código-fonte acessível ao público,
geralmente em um repositório na Internet, onde desenvolvedores e usuários
podem interagir.
%
O código é necessariamente licenciado sob termos legais formais que estão de
acordo com as definições da \textit{Free Software Foundation}
\footnote{\url{http://www.gnu.org/philosophy/free-sw.html}} ou da
\textit{Open Source Initiative}
\footnote{\url{http://www.opensource.org/docs/definition.html}}.

%------------------------------------------------------------------------------%

Uma vantagem oferecida pelo software livre em comparação ao software
restrito vem do fato de que o código-fonte pode ser livremente compartilhado.
%
Esse compartilhamento pode simplificar o desenvolvimento de aplicações
personalizadas, que não precisam ser programadas a partir do zero, mas
podem basear-se em soluções já existentes.
%
Na medida em que o desenvolvimento de aplicações personalizadas é um dos focos do
desenvolvimento de software em geral, essa vantagem tem impacto significativo na
redução de custos e na diminuição na duplicação de esforços, tirando proveito da
característica abstrata do software.

Outra vantagem resultante do compartilhamento do código se refere
à possível melhoria na qualidade \citep{CatedralBazzar}, em particular frente aos
problemas inerentes à sua complexidade, que é o que estamos interessados
nesta tese.
%
Isso se deve ao maior número de desenvolvedores e usuários envolvidos
com o software. Em outras palavras, um número maior de desenvolvedores, com diferentes
perspectivas e necessidades, é capaz de identificar melhorias e corrigir
mais \emph{bugs} em menos tempo e, consequentemente, promover refatorações que,
geralmente, levam à melhoria do código.

%
Além disso, um número maior de usuários gera situações de uso e
necessidades mais variadas, o que se traduz em um maior número
de \emph{bugs} identificados e mais sugestões de melhorias.

%------------------------------------------------------------------------------%

\section{Desenvolvimento de Software Livre}

O desenvolvimento de software livre possui características distintas do
modelo restrito. Portanto, a relação com o mercado, o processo de
desenvolvimento e o produto a ser oferecido devem ser abordados de maneiras
distintas.

No artigo \emph{A Catedral e o Bazar} \citep{CatedralBazzar}, Eric Raymond
levanta os aspectos que contribuem para que um projeto de software
livre tenha sucesso. Suas observações formaram a base do movimento pelo
código aberto por ele iniciado na virada do século.
% 
Ao observar o modelo de desenvolvimento do Linux, Raymond vislumbrou semelhanças
com um barulhento Bazar, onde centenas ou milhares de desenvolvedores davam a
sua contribuição que era, então, gerenciada por um pequeno grupo ou por
um ``ditador benevolente'' que dava orientações sobre a qualidade das
modificações propostas e as aceitava ou não.
%
Ao identificar essas características, Raymond aplicou essas práticas de
desenvolvimento do modelo Bazar de forma consciente em um projeto sob sua
coordenação, o Fetchmail. Ele elencou as principais características desse modelo,
típicas em boa parte dos projetos de software livre da atualidade, como sendo:

\begin{itemize}

%1
\item \emph{Bons programas nascem de necessidades pessoais.} Um projeto tem maiores chances de sucesso
    quando o desenvolvedor principal ou grupo de desenvolvedores principais tem interesse e sentem
    a necessidade pessoal de utilizar aquele software.

%2
\item \emph{Bons programadores sabem escrever bom código, mas excelentes
    programadores sabem reescrever e reutilizar código.} Raymond menciona a ``preguiça construtiva''
    como a ideia de que não se deve reinventar a roda, mas sim reaproveitar o que já existe e,
    se for o caso, modificar o que já existe para melhorá-lo.

%3
\item \emph{Esteja preparado para jogar fora código-fonte se necessário e começar de novo.} 
	Ou seja, dificilmente acerta-se na primeira vez.

%4	
\item \emph{Os usuários devem ser tratados como co-desenvolvedores.} Esse é o melhor caminho para 
    o aprimoramento do código e depuração eficaz.

%5  
\item \emph{Libere código cedo e libere frequentemente e ouça seus usuários.}
	Um erro comum de pessoas e grupos que se 
	iniciam no mundo do software livre é achar que seu software ainda não está pronto para ser
	liberado, que agora ainda não é o momento certo para se fazer isso. Segundo Raymond, o quanto
	antes o código for liberado e quanto maior a frequência de liberação de novas versões, melhor
	será o retorno obtido dos usuários e a possibilidade de angariar contribuidores para o projeto.

%6	
\item \emph{Dados olhos suficientes, todos erros são triviais.} Raymond chamou essa frase de Lei de Linus.
    Com milhares de pessoas lendo o código-fonte do Linux, os eventuais erros eram localizados e
    reportados muito rapidamente. Da mesma forma, com centenas de pessoas com conhecimento técnico
    para resolver aqueles erros, rapidamente aparecia um voluntário com a solução do problema.

%7
\item \emph{Trate seus testadores das versões Beta como um recurso valioso e eles logo se tornarão
	um recurso valioso.} Não há nada mais eficaz para encontrar problemas num programa e sugerir
	melhorias em suas funcionalidades do que um grupo de usuários ativos e motivados querendo utilizar
	esse programa e testar as novas funcionalidades o quanto antes.

%8	
\item \emph{A perfeição (em projetar) é alcançada não quando não há mais nada a adicionar, 
	mas quando não há nada para jogar fora.} Essa é uma ideia quase que de consenso entre os grandes
	cientistas e engenheiros. Deve-se buscar sempre as soluções mais simples.

%9
\item \emph{A melhor coisa depois de ter boas ideias é reconhecer as boas ideias de seus contribuidores.} 
    Um bom líder de um projeto de software livre não é necessariamente
    aquele que tem ótimas ideias, mas sim aquele que é capaz de criar o ecossistema de 
    colaboração que permita que as boas ideias emerjam e sejam valorizadas e adotadas.

\end{itemize}

Baseado nessas nove características, queremos atribuir a Eric Raymond, e 
discutir nesta tese, uma extensão do que ele chamou de Lei Linus:
\textit{Dados olhos e desenvolvedores suficientes, todos os bugs serão triviais e 
o código estará organizado para receber contribuições} -- que vamos chamar aqui
de ``Lei de Raymond''.

%------------------------------------------------------------------------------%

\section{Software livre como método de desenvolvimento de software}

Devido às atividades de produção de código, documentação, relatos de \textit{bugs}
entre outras, as comunidades de software livre vêm construindo coletivamente
sistemas de software reconhecidamente de qualidade, em
um ambiente de colaboração constante para atualização e evolução desses
sistemas, organizados na forma de um rossio~\citep{simon:08}.
% TODO: EXPLICAR Rossio...
Nesse contexto, os usuários não necessariamente restringem-se a ser apenas
agentes passivos, mas podem atuar como colaboradores ou produtores do software
que usam.

Esse fenômeno de produção coletiva extrapolou o movimento do software
livre, especialmente a partir da primeira década do Século 21, com o surgimento de 
serviços criados e mantidos pelos próprios usuários na Internet, como a
Wikipedia, blogs pessoais, canais de TV e rádios online~\citep{wikinomics}.
%
Tais serviços, somados às redes
sociais, fazem com que as pessoas realmente
acreditem que podem influenciar outras através de seus próprios meios de
comunicação~\citep{castells:06}.
%
Esse cenário, em que não fica clara uma diferenciação entre consumidor e produtor de
informação e, no caso do software, usuário e desenvolvedor, pode ser chamado de
\emph{cultura livre}.

No caso do software livre, essa cultura tipicamente se pauta nas
questões técnicas e se organiza como uma meritocracia, onde
valoriza-se fortemente as colaborações feitas para o projeto.
%
No entanto, as questões éticas, quando se apresentam, em geral são consideradas
tão importantes quanto as técnicas, já que o envolvimento com um projeto é
também determinado pelo interesse pessoal.
%
Pessoas e organizações que permanecem no longo prazo colaborando de alguma maneira com o
desenvolvimento de um software livre realmente acreditam que estão fazendo a 
diferença e ajudando o mundo de alguma forma, e essa motivação
faz com que sua dedicação seja diferenciada.
%
Mesmo no caso (cada vez mais comum) de desenvolvedores pagos para trabalhar em um projeto
livre, os aspectos éticos e o relacionamento com a comunidade
norteiam sua participação.

\begin{comment}
%TODO: Carlos Denner:
Acho que voce deveria incluir exemplos ilustrativos para todos os pontos que argumenta (e.g., "Pessoas e organizações que permanecem no longo prazo colaborando de alguma maneira com o desenvolvimento de um software livre realmente acreditam que estão fazendo a diferença e ajudando
o mundo de alguma forma"; "colaboram com um software livre para melhor reaproveitar o conhecimento produzido coletivamente, bem como atingir numa escala maior seu mercado consumidor"; "muitos projetos de software livre não vão além dos estágios iniciais de planejamento"; etc. etc.).
\end{comment}

Na prática, o desenvolvimento de software livre envolve administrar ou participar
de uma equipe de desenvolvimento onde não há necessariamente uma hierarquia formal,
mecanismos de pressão para o cumprimento de prazos e/ou grande formalismo em processos.
%
Uma das estratégias para atrair contribuidores e evitar que eles abandonem o
projeto é garantir a qualidade do código,
o que favorece a criação de um círculo virtuoso em que o código promove o crescimento
da comunidade e a comunidade ativa promove melhorias no código (esse é ponto central
discutido no estudos apresentados no Capítulo 4 desta tese).

Na busca por uma formalização sobre qual é a metodologia das comunidades de software
livre, estudos mostram que métodos ágeis e software livre têm formas de trabalhos
semelhantes. Por exemplo, o desenvolvimento de software livre é considerado um método ágil
por Martin Fowler~\citep{Fowler00orig}.
%
Alguns autores afirmam que o desenvolvimento de software livre é um método
ágil~\citep{Warsta2002} e apontam fortes semelhanças entre métodos ágeis e
software livre~\citep{Warsta2003}.
%
Por exemplo, há um mapeamento entre as práticas comuns usadas pelas
comunidades de software livre e equipes ágeis~\citep{corbucci:2011}.
%
Conceitualmente, os valores semelhantes são:

\begin{itemize}

\item {Indivíduos e interações são mais importantes que processos e ferramentas.}

\item {Software em funcionamento é mais importante que documentação a\-bran\-gen\-te.}

\item {Colaboração com o cliente (usuários) é mais importante que negociação de contratos.}

\item {Responder às mudanças é mais importante que seguir um plano.}

\end{itemize}

Além disso, várias práticas disseminadas pelas metodologias ágeis são usadas no
dia-a-dia dos desenvolvedores e equipes das comunidades
de software livre~\citep{corbucci:2011}:
(i) Código compartilhado (coletivo);
(ii) Projeto simples;
(iii) Repositório único de código;
(iv) Integração contínua;
(v) Código e teste;
(vi) Desenvolvimento dirigido por testes, e
(vii) Refatoração.

Observar e entender esses aspectos nos projetos de software livre tornam-se
relevantes à medida que muitos projetos de software livre não vão além dos
estágios iniciais e muitos acabam sendo abandonados antes de produzir
resultados razoáveis.
%
Isso sugere que, mesmo com o sucesso de alguns projetos de software livre,
as comunidades, com ou sem a participação de empresas, podem avançar no
acompanhamento do desenvolvimento dos projetos de software livre que participam.

Para ilustrar esse cenário, podemos observar alguns dados extraídos do
SourceForge.net, um dos mais populares repositórios de projetos de software livre.
%
Entre os seus 201.494 projetos cadastrados, 60.642
lançaram mais de uma versão, 40.228 foram baixados mais de uma vez, 23.754 têm
mais de um membro, e
apenas 12.141 projetos satisfazem esses três
critérios de seleção juntos. Isso sugere que não mais que 6\% dos
projetos no SourceForge.net foram capazes de constituir uma comunidade de usuários
e desenvolvedores que se beneficiem do estilo de desenvolvimento
Bazar~\citep{CatedralBazzar}.

De maneira geral, pode-se afirmar que há uma grande disposição para a criação de
projetos de software livre, mesmo que muitas das iniciativas ``falhem''.
%
Olhar o processo de desenvolvimento de software livre do ponto de vista da 
Engenharia de Software e as possíveis sinergias com os métodos ágeis podem
contribuir para um melhor rendimento dessa disposição na criação e colaboração
em torno de projetos de software livre.

Na prática, dentro do processo de desenvolvimento de software livre, após lançar
uma versão inicial e divulgar o projeto, os usuários interessados começam a
usar o software livre em questão. 
%
Lembrando o que foi salientado por Eric Raymond sobre ``bons programas nascerem de
necessidades pessoais'', esses usuários podem também ser desenvolvedores, que
irão colaborar com o projeto a fim de atenderem às suas próprias necessidades.
%
Destacando a colaboração no código-fonte, essas melhorias são enviadas aos
mantenedores do projeto como \emph{patches}, ou seja, arquivos que
contém as modificações no código e que serão analisados pelos mantenedores que,
caso concordem com a mudança e com a sua implementação em si, irão
aplicá-las ao repositório oficial do projeto.

Mesmo que em projetos maiores outros aspectos sejam levados em consideração ou
sigam processos mais burocrático de colaboração, a essência da colaboração
técnica está no envio e análise de trechos de código-fonte.
%
Neste contexto, nesta tese de doutorado, argumentamos e discutimos, através de
estudos baseados em análises estatísticas, que o monitoramento de métricas de
código-fonte, como sendo uma das atividades de acompanhamento (do mesmo conceito de
``tracking'' dos métodos ágeis), pode ser um recurso  valioso de apoio
ao gerenciamento da evolução dos projetos de software livre com o objetivo de
atrair e facilitar a colaboração dos desenvolvedores.

