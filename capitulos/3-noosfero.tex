\chapter{Noosfero}


\section{Software Livre}

Software expressa uma solução abstrata dos problemas computacionais.
%
O software, em um sistema computacional, é o componente que contém o
conhecimento relacionado aos problemas a que a computação se aplica.
%
Por isso, o software é algo de interesse geral, uma vez que vários aspectos
relacionados a ele ultrapassam as questões técnicas, como por exemplo:
\begin{itemize}
\item O processo de desenvolvimento do software; 
\item Os mecanismos econômicos (gerenciais, competitivos, sociais, cognitivos etc.)
que regem esse desenvolvimento e seu uso;
\item O relacionamento entre desenvolvedores, fornecedores e usuários do
        software;
\item Os aspectos éticos e legais relacionados ao software.
\end{itemize}

O que define e diferencia o software livre do que podemos denominar de
software restrito passa pelo entendimento desses quatro pontos dentro do que é
conhecido como o \emph{ecossistema do software livre}.
%
O princípio básico desse ecossistema é promover a liberdade do usuário,
sem discriminar quem tem permissão para usar um software e seus limites de uso,
baseado na colaboração e num processo de desenvolvimento aberto.
%
Software livre é aquele que permite aos usuários usá-lo, estudá-lo, modificá-lo e
redistribui-lo, em geral, sem restrições para tal e prevenindo que não sejam
impostas restrições aos futuros usuários.
%

Normalmente, esse software existe por meio de projetos de desenvolvimento
que estão centradas em torno de algum código-fonte acessível ao público,
geralmente em um repositório na Internet, onde desenvolvedores e usuários
podem interagir.
%
O código é necessariamente licenciado sob termos legais formais que estão de
acordo com as definições da \textit{Free Software Foundation}
\footnote{\url{http://www.gnu.org/philosophy/free-sw.html}} ou da
\textit{Open Source Initiative}
\footnote{\url{http://www.opensource.org/docs/definition.html}}.

%------------------------------------------------------------------------------%

Uma vantagem oferecida pelo software livre em comparação ao software
restrito vem do fato de que o código-fonte pode ser livremente compartilhado.
%
Esse compartilhamento pode simplificar o desenvolvimento de aplicações
personalizadas, que não precisam ser programadas a partir do zero, mas
podem basear-se em soluções já existentes.
%
Na medida em que o desenvolvimento de aplicações personalizadas é um dos focos do
desenvolvimento de software em geral, essa vantagem tem impacto significativo na
redução de custos e na diminuição na duplicação de esforços, tirando proveito da
característica abstrata do software.

Outra vantagem resultante do compartilhamento do código se refere
à possível melhoria na qualidade \cite{CatedralBazzar}, em particular frente aos
problemas inerentes à sua complexidade.
%
Isso se deve ao maior número de desenvolvedores e usuários envolvidos
com o software. Em outras palavras, um número maior de desenvolvedores, com diferentes
perspectivas e necessidades, é capaz de identificar melhorias e corrigir
mais \emph{bugs} em menos tempo e, consequentemente, promover refatorações que,
geralmente, levam à melhoria do código.

%
Além disso, um número maior de usuários gera situações de uso e
necessidades mais variadas, o que se traduz em um maior número
de \emph{bugs} identificados e mais sugestões de melhorias.

%------------------------------------------------------------------------------%

\section{Noosfero}

