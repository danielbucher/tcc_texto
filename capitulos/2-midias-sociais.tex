\chapter{Mídias Sociais na Educação}
\label{cap:midias-sociais}
%/ref{cap:midias-sociais}

\section{Mídias Sociais}

%Traduzido da pág. 1 do texto base
A mídia social como uma tecnologia se tornou um fenômeno em crescimento com
diversas definições para o público e para uso acadêmico. No geral, mídia social
se refere à mídias utilizadas para possibilitar a interação social. No contexto
desse trabalho, tecnologias de mídias sociais (SMT - Social Media Technology)
diz respeito a aplicações web e mobile que permitem que indivíduos e organizações
criem e compartilhem novos conteúdos gerados pelo usuário, ou conteúdos já
existentes, em ambiente digital através de comunicação em várias vias. É
importante notar a diferença entre conteúdo gerado pelo usuário, que é uma forma
de mídia não tradicional desenvolvida e produzida por usuários, e conteúdo já
existente, que em geral se refere à mídia tradicional (jornais, revistas, rádios
e televisão) reproduzida para a web. Além destas características, as SMT também
contêm elementos de design que criam espaços sociais virtuais que encorajam a 
interação, ampliam o apelo da tecnologia e promovem transições nos dois sentidos
entre a interação através da plataforma e a interação cara a cara \cite{davis2012}.

%Traduzido da pág. 1 do texto base
%TODO: Revisar importância dessa parte
O uso de tecnologias de mídias sociais através de computadores e dispositivos móveis
se tornou bastante difundida, principalmente através do Facebook e do Twitter.
O primeiro funciona como uma rede social que permite que seu usuário crie um 
perfil virtual através do qual este realiza interações sociais com outros perfis,
expressa seus interesses e descobre aspectos comuns com outros usuários. O conceito
de redes sociais é explicado na próxima seção. O segundo funciona como uma
interface através da qual um usuário consegue compartilhar conteúdo limitado de
forma fácil com um número extensivo de outros usuários sem a necessidade de criar
e manter um perfil.
%Traduzido e adaptado da pág. 2 do texto base
As aplicações de mídias sociais compartilham da habilidade natural de viabilizar
comportamento social através do diálogo - discussões de múltiplas vias que
fornecem a oportunidade de descobrir e compartilhar informação nova
\cite{solis2008}.
%
Portanto, as SMT são um terreno vasto como software com possibilidades de uso
variadas, não estando limitada a redes sociais, compartilhamento de vídeo ou
blogs. Uma definição das SMT mais abrangente seria a totalidade de produtos
digitais e serviços disponibilizados online, comportamento social e troca de
conteúdo que possuem como fonte principalmente o usuário.

A definição de SMT acima não contempla, no entanto, plataformas de ensino como o
Moodle, que possuem um foco em propósitos educacionais específicos e não foram
criadas para ter nos usuários sua principal fonte de conteúdo. Dificilmente
vemos em uma plataforma como o Moodle, um estudante iniciando um tópico cujo
propósito seja compartilhar conteúdo com os demais. Em geral, a comunicação
realizada via Moodle é vertical, parte do professor para os alunos.

\subsection{A Difusão das Mídias Sociais}

%referenciar
Com a proliferação das redes sociais e outras plataformas de mídia social nos
últimos anos, a pervasividade da internet se tornou mais evidente do que nunca.
Levamos aspectos de nossas vidas pessoais, nossos pensamentos políticos, nossas
experiências profissionais, dentre outros, para a internet. Diferente de outras
tecnologias de comunicação na internet, as SMT nos forneceram um ambiente
virtual que nos remete a elementos de comunidade vivenciados fora da internet
\cite{davis2012}.

%traduzido e adaptado da pág. 3 do texto base
As tecnologias de mídia social conectam as pessoas de forma similar aos sentimentos
tradicionais de conexão, de pertencer a um grupo, de intercâmbio de sentimentos
e ideias.As SMT diminuíram o custo para se colaborar, compartilhar e produzir,
assim fornecendo novas e revolucionárias formas de resolver problemas. Agora
podemos manter e acessar comunidades online e ao mesmo tempo utilizar as SMT
como ferramentas para transitar entre o contato online e o contato cara a cara
através de amizades, atividades planejadas e eventos marcados \cite{shirky2010}.

%traduzido e adaptado da pág. 4 do texto base
A geração de alunos de graduação atual abraçou as mídias sociais . Ela se tornou
uma parte importante de seu cotidiano. As fronteiras entre comunidades online e
comunidades do "mundo real" estão se estreitando, se não deteriorando
completamente. Se considerarmos a geração que não conheceu um mundo sem as
tecnologias de mídia social, existe um intercâmbio contínuo entre o experiências
físicas e digitais. Para essa geração, as SMT são a forma de comunicação e de
busca de informação principal e possivelmente um componente central na construção
de suas identidades.
%traduzido e adaptado da pág. 4 do texto base
Em 2008, em uma entrevista, o Professor de Psiquiatria da Universidade da Califórnia,
Los Angeles, Dr. Gary Small, sugeriu que — jovens que nasceram em um mundo de laptops
e celulares, mensagens de textos e tweets — gastam, em média, mais de oito horas
por dia expostas à tecnologia digital \cite{lin2008}. 
%
Como resultado, esses nativos da era digital podem experienciar um
desenvolvimento do cérebro fundamentalmente diferente que favorece a comunicação
constante e a multi-tarefa \cite{prensky2001} \& \cite{vorgan2009}.

%copiado e adaptado do texto do Google Drive.
%referenciar
Desse ponto de vista, não é preciso ir muito longe para deduzir as SMT estão
transformando também a forma com que as comunidades de estudantes se comunicam
entre si. As tecnologias de mídia social também afetaram as universidades, de
forma que o número de perfis ou páginas oficiais de universidades é cada vez mais
comum, chegando a atingir 100\% das instituições de graduação americanas. As
universidades passaram a utilizar essas tecnologias como forma de promover uma
interação maior com seus integrantes e como forma de divulgar seus projetos de
pesquisa. O Community College Survey of Student Engagement constatou
que alunos que utilizam tecnologias de mídia social com o propósito de realizar
atividades acadêmicas possuem maior nível de engajamento nelas \cite{ccsse2009}.
No entanto, a maioria das plataformas de SMT utilizadas por esses alunos e
universidades são as mesmas que tendem a querer centralizar as atividades nelas
e , em geral, a interação social entre seus usuários não costumam fornecer
mecanismos específicos para o intercâmbio de conhecimento científico e acadêmico.
%
Além disso, a relação entre alunos e Universidades por meio das SMT costuma
permanecer restrita à divulgação de notícias, comunicados e divulgação de
conquistas por parte da instituição de ensino, de certa forma ainda uma
comunicação vertical, permanecendo, em geral, fora da sala de aula e dos
laboratórios.

\subsection{O Potencial das Mídias Sociais na Educação}

A metodologia de ensino tradicional, fundada na transmissão de conhecimento de
"A" para "B", ou seja, do professor para o aluno, permanece alheia ao movimento
das novas tecnologias comunicacionais e ao perfil do novo aluno. Nesse aspecto
as SMT possuem o potencial propiciar a quebra da verticalização na relação
professor/aluno. O professor que busca interatividade com seus alunos passaria
atuar de forma a propor um conhecimento ao invés de transmiti-lo,  atuando mais
como facilitador e guia do que como um transmissor \cite{silva2002}.

As redes sociais, uma das camadas das SMT, se adaptam bem a essa forma de
ensino. O professor seria responsável por disponibilizar comunidades para
diferentes temas ou disciplinas e os alunos seriam encorajados a interagir com
essas comunidades e criar conteúdo para elas de forma colaborativa. Ele passa
a não estar mais reduzido a olhar, ouvir e copiar o conteúdo, mas sim a criá-lo,
distribuí-lo, tornando-se assim um co-autor daquele conteúdo e adquirindo a
autoconfiança necessária para desenvolverm seu potencial. E o professor
atuaria como um guia e um incentivador.

\section{Redes Sociais}

É comum vermos a utilização do termo "redes sociais" para se dirigir a todos os
tipos de mídias sociais mediadas por computador, no entanto vale ressaltar que,
embora muito relevante, redes sociais são apenas uma das camadas das mídias sociais.

Ellison e Boyd definem sites de redes sociais como serviços web que permitem que
seus usuários criem perfis, através desses perfis, conexões com outros usuários,
busquem e cruzem informação dentro dessa lista de conexões \cite{beer2008}.

%Parágrafo copiado do texto introdutório do TCC no google docs.
Por outro lado, já existem casos em universidades, incluso no Brasil, de redes 
usadas com o objetivo de promover a interação em torno das colaborações em si e
não das pessoas, ou seja, expandindo para o conceito de redes de colaboração. As
pessoas entram na rede para fazerem parte e acompanharem uma disciplina, um projeto
ou um determinado grupo de trabalho da universidade. Nesse cenário, o professor
ao divulgar um curso multidisciplinar, pode usufruir das estatísticas e
comportamentos das pessoas na rede para chegar ao seu público alvo. Da mesma
forma, alunos podem encontrar projetos e grupos de trabalhos de seu interesse ao
explorar a rede, com a ajuda da própria rede de colaboração. Argumentamos que
esse tipo de dinâmica não é possível em redes centralizadoras e monopolistas
porque nelas o conteúdo, em geral, é pulverizado (e em alguns casos controlado).
Isso, somado ao fato da característica principal da Internet ser uma "rede de redes",
faz com que as redes sociais e de colaboração tendam a serem melhor utilizadas
em um nicho específico e com autonomia para seus gestores e usuários.

Nessa seção são apresentados dois exemplos da utilização de redes sociais no
contexto da educação. Vale ressaltar que, em ambos os casos, as redes sociais
foram criadas através de plataformas livres. A plataforma para redes sociais
livres Noosfero e a importância de se utilizar plataformas livres para a
implantação de redes sociais em ambientes educacionais são abordadas no capítulo
três desse texto.

\subsection{Stoa}

Baseado nas ideias discutidas anteriormente, a Universidade de São Paulo criou o
projeto Stoa~\footnote{\url{social.stoa.usp.br}}, uma rede de colaboração e
disseminação do conhecimento apoiada por três princípios: o compartilhamento, a
liberdade e a horizontalidade.
%
A rede Stoa permite ao usuário a criação de seu espaço pessoal e a liberdade de
publicar suas ideias, ou o conteúdo que desejar, por exemplo, na forma de
\textit{blogs} pessoais, \textit{blogs} de disciplinas, pesquisas em andamento,
dentre outras, além de compartilhar esse conteúdo para ser acessível
para outros usuários da rede (e fora da rede).


O Stoa foi lançado em 2007 baseado da plataforma de software livre
Elgg~\footnote{\url{elgg.org}}.
%
Por volta de 2010, começou-se a perceber algumas limitações das tecnologias
utilizadas e foram feitas pesquisas para levantar alternativas que permitissem
implantar uma rede social de colaboração com qualidade e que fosse compatível
com conceitos da chamada Web 2.0.
%
Foi optado pelo Noosfero~\footnote{\url{noosfero.org}}, uma plataforma para
criação de rede sociai e de economia solidária, livre, desenvolvida pela
Empresa Cooperativa Colivre \footnote{\url{colivre.coop.br}}, por este ter um
grande potencial devido a suas funcionalidades avançadas que permitem a criação
e o compartilhamento de conteúdo de forma satisfatória e pela posição geográfica
que permite uma aproximação privilegiada do núcleo desenvolvedor do mesmo.
%
Em dezembro de 2012, a USP lançou a nova versão da rede social do Stoa, baseado
no Noosfero, e a chamando de rede de colaboração.

O Stoa não se propõe a ser a única rede de colaboração acadêmica no Brasil. A
proposta de seus idealizadores é que a experiência na Universidade de São Paulo
possa ser replicada em outras instituições de ensino brasileiras.
%
A rede do Stoa tem o objetivo de ser uma rede dentre outras redes, mas com a ideia de poder
interagir com as demais redes. Outro aspecto fundamental para a escolha do Noosfero
como plataforma se deu, entre outros aspectos, por conta da possibilidade e
planejamento dessa plataforma poder suporta a federação entre as redes (baseadas
no Noosfero ou em outras plataformas que implementem um protocolo aberto e comum).
%
O termo redes sociais federadas é usado para indicar uma rede social autônoma
controlada por uma entidade mas que possibilita a interação, através de regras
acordadas, com usuários ou entidades de outras redes federadas sem a necessidade
de criar uma conta na segunda \cite{prodomou2010}.

\subsection{TecCiência}

Outro caso interessante do uso de redes sociais na educação surgiu através do
projeto EDUCANDOW \cite{santos2012} - Educação em Ciência e Tecnologia para
Escolas de Ensino Fundamental do Município Candeias. O projeto EDUCANDOW surgiu
em 2007 através de uma parceria entre a empresa Dow Brasil S.A. e o Departamento
de Ciência da Computação da Universidade Federal da Bahia com o objetivo de
promover uma educação básica qualificada e consistente com as tendências tecnológicas
do momento. Como parte do projeto, em 2009 foi implantada uma rede social como
ambiente educacional, denominada TecCiencia \footnote{\url{tecciencia.ufba.br}}
na escola SESI de Candeias, Bahia.

A equipe do projeto EDUCANDOW buscou experimentar uma perspectiva educacional que
alinhe a educação às tecnologias contemporâneas.
%
Isso porque, a prática pedagógica coerente com a sociedade atual implica no uso
constante de ferramentas digitais e na educação em rede para quebrar a verticalização
da relação professor/aluno \\ \cite{santos2012}.
% \\ força quebra de linha...

Resolveram então apostar no software livre por entender que este é um integrante
fundamental no ensino tecnológico e também adotaram a plataforma Noosfero por sua
capacidade de ser facilmente adaptado para as mais variadas necessidades. O
TecCiencia apresenta-se como um ambiente de aprendizado interativo que volta-se
para a organização desse aprendizado através da disponibilização de comunidades
que permitem aos estudantes a construção do conhecimento a medida que são
desafiados à buscar soluções para problemas de forma colaborativa.

Para permitir o uso adequado do potencial da ferramenta, os professores da escola
SESI de Candeias receberam treinamento na utilização da ferramenta e de como alinhar
seu uso à uma perspectiva construtivista de ensino. Além disso, foi elaborado um
material didático utilizando metodologias modernas de ensino como webquests e
rádio web.

O TecCiencia apresenta-se como um ambiente de interatividade voltado para a
organização da aprendizagem, onde as orientações didáticas estão disponibilizadas
através de comunidades que possibilitam aos estudantes a construção de seus
conhecimentos, a medida que são desafiados à buscar soluções para determinadas
tarefas e construções coletivas. O ambiente contribui para que os alunos encontrem
suas próprias fontes incentivando a ampliação da aprendizagem autônoma e a
troca de ideia entre grupos ou indivíduos.

Assim como no caso do Stoa, o TecCiencia não se propõe a substituir o aprendizado
em sala de aula, mas sim a acrescentar no processo de aprendizagem aspectos que
dificilmente são conseguidos através da abordagem vertical tradicional. Para
estimular a utilização da plataforma, foram utilizados recursos como gincanas
virtuais e a confecção de um jornal virtual.A avaliação dos alunos envolvidos no
projeto eram realizadas com foco na interação dos alunos para construir e expandir
conhecimento, solucionar problemas e na atuação e disponibilização de conteúdo
em seus espaços virtuais.

Foram observados diversos resultados positivos ao final de dois anos de
experiência, sendo o mais imediato foi o reconhecimento de uma aproximação maior
entre os professores e os alunos. Outros resultados observados foram a
capacitação dos professores e a melhoria das notas da turma do 8º ano B de 2010. 
