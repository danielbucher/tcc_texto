%TODO rever referencias
\chapter{Mídias Sociais}
\label{cap:midias-sociais}

A mídia social como uma tecnologia se tornou um fenômeno em crescimento com
diversas definições para o público~\cite{davis2012}. No geral, mídia social
se refere às mídias utilizadas para possibilitar a interação social. No contexto
desse trabalho, tecnologias de mídias sociais (SMT - \textit{Social Media Technology})
diz respeito a aplicações web e mobile que permitem que indivíduos e organizações
criem e compartilhem novos conteúdos gerados pelo usuário, ou conteúdos já
existentes, em ambiente digital através de comunicação em várias vias~\cite{davis2012}. É
importante notar a diferença entre conteúdo gerado pelo usuário, que é uma forma
de mídia não tradicional desenvolvida e produzida por usuários, e conteúdo já
existente, que em geral se refere à mídia tradicional (jornais, revistas, rádios
e televisão) reproduzida para a web. Além destas características, as SMT também
contêm elementos de design que criam espaços sociais virtuais que encorajam a 
interação, ampliam o apelo da tecnologia e promovem transições nos dois sentidos
entre a interação através da plataforma e a interação cara a cara~\cite{davis2012}.

As aplicações de mídias sociais compartilham da habilidade natural de viabilizar
comportamento social através do diálogo - discussões de múltiplas vias que
fornecem a oportunidade de descobrir e compartilhar informação nova
\cite{solis2008}.
%
Portanto, as SMT são um terreno vasto como software com possibilidades de uso
variadas, não estando limitada a redes sociais, compartilhamento de vídeo ou
blogs. Uma definição das SMT mais abrangente seria a totalidade de produtos
e serviços digitais, disponibilizados online; o comportamento social e troca de
conteúdo que possuem como fonte principalmente o usuário \cite{davis2012}.

\section{A Difusão das Mídias Sociais}

Com a proliferação das redes sociais e outras plataformas de mídia social nos
últimos anos, a pervasividade da internet se tornou mais evidente do que nunca~\cite{davis2012}.
%
Levamos aspectos de nossas vidas pessoais, nossos pensamentos políticos, nossas
experiências profissionais, dentre outros, para a internet. Diferente de outras
tecnologias de comunicação na internet, as SMT nos forneceram um ambiente
virtual que nos remete a elementos de comunidade vivenciados fora da internet
\cite{davis2012}.

As SMT diminuíram o custo para se colaborar, compartilhar e produzir,
assim fornecendo novas e revolucionárias formas de resolver problemas~\cite{shirky2010}.
%
Agora podemos manter e acessar comunidades online e ao mesmo tempo utilizar as SMT
como ferramentas para transitar entre o contato online e o contato cara a cara
através de amizades, atividades planejadas e eventos marcados \cite{shirky2010}.

As mídias sociais tornaram-se  uma parte importante do cotidiano dos atuais jovens.
É cada vez mais difícil separar as relação em comunidades virtuais das relações em
comunidades ``reais''~\cite{davis2012}.
%
Se considerarmos a geração que não conheceu um mundo sem as
tecnologias de mídia social, existe um intercâmbio contínuo entre o experiências
físicas e digitais.~\cite{davis2012}.
%
Como um possível resultado disso, esses nativos da era digital podem experienciar um
desenvolvimento do cérebro fundamentalmente diferente que favorece a comunicação
constante e a multi-tarefa \cite{prensky2001} \& \cite{vorgan2009}.

\section{Redes Sociais}

É comum vermos a utilização do termo ``redes sociais'' para se dirigir a todos os
tipos de mídias sociais mediadas por computador, no entanto vale ressaltar que,
embora muito relevante, redes sociais são apenas uma das camadas das mídias sociais.
%
\apudonline{boyd2007}{beer2008} definem sites de redes sociais como serviços
web que permitem que seus usuários criem perfis, através desses perfis,
conexões com outros usuários, busquem e cruzem informação dentro dessa lista
de conexões.

No contexto deste trabalho estamos tratando, especificamente, redes sociais como redes 
usadas com o objetivo de promover a interação em torno das colaborações em si e
não das pessoas, ou seja, expandindo para o conceito de redes de colaboração, o que
se aplicar ao contexto de uma rede de nicho de uma universidade. 
%
Por exemplo, as pessoas entram na rede para fazerem parte e acompanharem uma disciplina, um projeto
ou um determinado grupo de trabalho da universidade.
%
Nesse cenário, hipoteticamente, o professor ao divulgar um curso multidisciplinar, pode usufruir das estatísticas e
comportamentos das pessoas na rede para chegar ao seu público alvo.
%
Da mesma forma, alunos podem encontrar projetos e grupos de trabalhos de seu interesse ao
explorar a rede, com a ajuda da própria rede de colaboração.

Argumentamos que esse tipo de dinâmica não é possível em redes centralizadoras e monopolistas,
porque nelas o conteúdo, em geral, é pulverizado (e em alguns casos controlado).
Isso, somado ao fato da característica principal da Internet ser uma ``rede de redes'',
faz com que as redes sociais e de colaboração tendam a serem melhor utilizadas
em um nicho específico e com autonomia para seus gestores e usuários.

Não entrando no mérito do uso de rede sociais na educação, uma vez que não tratamos disto neste trabalho,
nessa seção apresentamos dois casos da utilização de redes sociais como rede de
colaboração para disseminação de conhecimento, para exemplificarmos algumas possibilidades de uso.
%
Em ambos os casos, tais redes foram criadas através de softwares livres, no caso, a  plataforma para 
redes sociais Noosfero, que apresentaremos em detalhes no capítulo~\ref{cap:noosfero} desse texto.

\subsection{Stoa}

Baseado nas ideias discutidas anteriormente, a Universidade de São Paulo criou o
projeto Stoa~\footnote{\url{social.stoa.usp.br}}, uma rede de colaboração e
disseminação do conhecimento apoiada por três princípios: o compartilhamento, a
liberdade e a horizontalidade.
%
A rede Stoa permite ao usuário a criação de seu espaço pessoal e a liberdade de
publicar suas ideias, ou o conteúdo que desejar, por exemplo, na forma de
\textit{blogs} pessoais, \textit{blogs} de disciplinas, pesquisas em andamento,
dentre outras, além de compartilhar esse conteúdo para ser acessível
para outros usuários da rede (e fora da rede).

O Stoa foi lançado em 2007 baseado da plataforma de software livre
Elgg~\footnote{\url{elgg.org}}.
%
Por volta de 2010, começou-se a perceber algumas limitações das tecnologias
utilizadas e foram feitas pesquisas para levantar alternativas que permitissem
implantar uma rede social de colaboração com qualidade e que fosse compatível
com conceitos da chamada Web 2.0.
%
Foi optado pelo Noosfero~\footnote{\url{noosfero.org}}, uma plataforma para
criação de rede sociai e de economia solidária, livre, desenvolvida pela
Empresa Cooperativa Colivre \footnote{\url{colivre.coop.br}}, por este ter um
grande potencial devido a suas funcionalidades avançadas que permitem a criação
e o compartilhamento de conteúdo de forma satisfatória e pela posição geográfica
que permite uma aproximação privilegiada do núcleo desenvolvedor do mesmo.
%
Em dezembro de 2012, a USP lançou a nova versão da rede social do Stoa, baseado
no Noosfero, e a chamando de rede de colaboração.

O Stoa não se propõe a ser a única rede de colaboração acadêmica no Brasil. A
proposta de seus idealizadores, com quem também interagimos durante este trabalho,
é que a experiência na Universidade de São Paulo possa ser replicada em
outras instituições de ensino brasileiras.
%
A rede do Stoa tem o objetivo de ser uma rede dentre outras redes, mas com a ideia de poder
interagir com as demais redes, em especial, quando plataforma de rede sociais suportarem
a denominada federação entre as redes~\footnote{O termo redes sociais federadas é usado 
para indicar uma rede social autônoma controlada por uma entidade mas que possibilita a
interação, através de regras acordadas, com usuários ou entidades de outras redes
federadas sem a necessidade de criar uma conta na segunda \cite{prodomou2010}}.
%
Por conta da pertinência da implementação de um protocolo aberto de federação no Noosfero,
na Seção~\ref{sec:future-works} , de trabalho futuros, apresentamos um breve relato de possíveis
protocolos que possam ser incorporados nessa plataforma.

\subsection{TecCiência}
\label{subsec:tecciencia}

Outro caso do uso de redes sociais de compartilhamento de conhecimento surgiu através do
projeto EDUCANDOW  - Educação em Ciência e Tecnologia para
Escolas de Ensino Fundamental do Município Candeias~\cite{santos2012}.
%
O projeto EDUCANDOW surgiu, em 2007, através de uma parceria entre a empresa
Dow Brasil S.A. e o Departamentode Ciência da Computação da Universidade Federal da Bahia,
com o objetivo de promover uma educação básica qualificada e consistente com as tendências tecnológicas
do momento.
%
Como parte do projeto, em 2009, foi implantada uma rede social, denominada TecCiencia~\footnote{\url{tecciencia.ufba.br}}
na escola SESI de Candeias, Bahia.

A equipe do projeto EDUCANDOW buscou experimentar uma perspectiva educacional que
alinhe a educação às tecnologias contemporâneas.
%
\citeonline{santos2012} defendem que a prática pedagógica coerente com a sociedade atual
implica no uso constante de ferramentas digitais e na educação em rede para quebrar a
verticalização da relação professor/aluno.

Resolveram então apostar no software livre por entender que este é um integrante
fundamental no ensino tecnológico e também adotaram a plataforma Noosfero por sua
capacidade de ser facilmente adaptado para as mais variadas necessidades. O
TecCiencia apresenta-se como um ambiente de aprendizado interativo que volta-se
para a organização desse aprendizado através da disponibilização de comunidades
que permitem aos estudantes a construção do conhecimento a medida que são
desafiados à buscar soluções para problemas de forma colaborativa~\cite{santos2012}.

Assim como no caso do Stoa, o TecCiencia provê um ambiente que permite a continuidade do
aprendizado em sala de aula, em um ambiente virtual compartilhado.
%
Nas instituições de ensino, por exemplo, suas estruturas físicas tentam equilibrar espaços ``formais'',
como as salas de aula, com ambientes públicos de relação social entre professores, alunos e funcionários,
como praças, cantinas, centros acadêmicos, entre outros. Da mesma forma, o Stoa e TecCiencia tentam
ser essa espaço aberto e público de colaboração no ``mundo virtual'', contraponto com, por exemplo, seus
sites institucionais.
%
Na UnB, o Comunidade.UnB terá este mesmo objetivo, através da implantação de uma
rede também baseada no Noosfero (que apresentaremos no próximo capítulo).
