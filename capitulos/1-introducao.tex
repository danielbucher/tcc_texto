\chapter{Introdução}
 
A utilização de redes sociais têm se tornado cada vez mais comum, em especial
para as gerações mais jovens. Neste contexto, não estamos nos
referindo apenas ao \textit{Facebook}, \textit{Orkut} e \textit{Twitter}. 
%
Por exemplo, atualmente, é bem comum empresas terem suas próprias redes sociais para
promoverem a interação entre os seus funcionários e colaboradores: como é o caso do
Você.Serpro~\footnote{\url{voce.serpro.br}, acessível apenas dentro da rede do Serpro:
\url{http://www.anaisdoconserpro.serpro.gov.br/modules/cadastro_de_trabalhos/trabalho.php?cod=219&ano=2012}}
%
A ideia de não depender de redes centralizadas, como é o caso das supracitadas, faz com que
as campanhas de Barack Obama~\footnote{\url{http://my.barackobama.com}}, nos
Estados Unidos, e Dilma Rousseff~\footnote{\url{http://dilmanarede.com.br}}, no
Brasil, sejam exemplos da necessidade de autonomia das suas redes
(e informações delas) e como melhor explorar a Internet, que tem em sua
concepção ser descentralizada. 
%TODO: Referenciar livro sobre mídias sociais...
Argumentamos neste trabalho que as tecnologias de mídia social exercem um papel importante para 
essa descentralização uma vez que possuem como característica a criação e o
compartilhamento de conteúdo de forma horizontal e colaborativa. 
%
Baseado nessa visão, a proposta desse trabalho é disponibilizar um ambiente
virtual que possibilite que os alunos da Universidade de Brasília criem e
compartilhem conhecimento de forma colaborativa e horizontal.
 
Estudos mostram que a maioria dos alunos de graduação faz uso de algum
tipo de rede social e que aqueles que participam de forma mais ativa delas
tendem a obter maior riqueza em suas relações sociais \cite{???}.  ***
%
Também, deve-se enfatizar que, está no cerne da criação das instituições de
ensino superior universalizar o conhecimento. Compartilhar e dar subsídios para
que o conhecimento seja disseminado e reproduzível é um dos pilares da ciência
\cite{Kon2011}.
%
%
Adaptando essas ideias para o ponto de vista tecnológico, sites de rede social
são apenas uma camada das tecnologias de mídia social. A definição mais ampla de
tecnologias de mídia social inclui: a totalidade de "produtos" e "serviços"
digitais disponibilizados online; o comportamento social gerado pelo usuário; e
a permutação de conteúdo gerado primariamente pelos próprios usuários.
\cite{???} ***
 
 
%------------------------------------------------------------------------------%
 
Não é enfoque deste trabalho, mas há um levantamento que as tecnologias de mídia social deram às
instituição de ensino a oportunidade de, por exemplo, divulgar as conquistas de seus alunos,
criando assim um sentimento de lealdade e ao mesmo tempo atraindo alunos em
potencial \cite{solis2008}. 
%
Além disso, foi constatado que alunos que utilizam tecnologias de mídia social
com o propósito de realizar atividades acadêmicas possuem maior nível de
engajamento nelas  \cite{ccsse2009}. 
 
%------------------------------------------------------------------------------%
 
Sob essa perspectiva das redes sociais no âmbito acadêmico, implantamos uma versão de homologação
de uma rede social própria para a Universidade de
Brasília, denominada de Comunidade.UnB~\footnote{\url{\http://comunidade.unb.br}}, baseada na ferramenta de software livre 
Noosfero~\footnote{\url{http://noosfero.org/}},
na qual os usuários poderão publicar e compartilhar conteúdo livremente e
colaborar com a difusão dessa nuvem de conhecimento e ideias que a Universidade
nos proporciona, sem perder a autonomia do mesmo ao, inclusive, definir os termos
de uso e licenças de tais conteúdos.
%
Também, colaboramos diretamente com a comunidade do Noosfero na implementação de
de um conjuntos de funcionalidades, de acordo com o levantamento de requisitos feito,
bem como da avaliação das melhorias sugeridas pela comunidade do Noosfero, que mantém
um \textit{issue tracker}~\footnote{\url{http://en.wikipedia.org/wiki/Issue_tracking_system}}
público em sua página de desenvolvimento~\footnote{\url{https://noosfero.org/Development/}}.
 
%------------------------------------------------------------------------------%
 
\section{Objetivos}
 
%TODO: rever!!!
\subsection{Objetivos Gerais}
 
Neste trabalho de conclusão de curso, implementamos as
principais funcionalidades para que uma rede social de colaboração, como o Stoa
~\footnote{\url{http://social.stoa.usp.br}} da USP, 
possa ser também disponibilizada na Universidade de Brasília. 
%
Dessa forma, colaboramos com o desenvolvimento dessa plataforma, interagindo diretamente com
a  comunidade de desenvolvedores do Noosfero, utilizando práticas ágeis e formas de trabalhar
com colaboradores presenciais e distribuídos.
 
 
%------------------------------------------------------------------------------%
 
\subsection{Específicos}
 
Os objetivos específicos desse trabalho foram:
 
\begin{enumerate}
    
\item Identificar as tecnologias utilizadas pela UnB através das quais será
possível autenticar-se na comunidade.
\item Implantar uma instância do Noosfero e disponibilizar para a comunidade.
\item Integrar a instância do Noosfero às tecnologias cabíveis.
\item Adequar a instância do Noosfero ao padrão visual da UnB.
\item Levantar junto aos estudantes um conjunto de funcionalidades a serem
encorporadas ao Noosfero para disponibilizar um ambiente virtual adequado ao
ensino.
\item Desenvolver novas funcionalidades selecionadas e melhorias (correções de defeitos) necessários.
 
\end{enumerate}
 
%------------------------------------------------------------------------------%
 
\section{Organização do Trabalho}
 
Durante a primeira fase do projeto, o trabalho foi conduzido na forma de
pesquisas de fundamentação teórica sobre o uso de mídias sociais.
Posteriormente, instalamos uma instância do Noosfero em um
servidor de testes disponibilizadas para nós pelo Centro de Difusão de
Tecnologia e Conhecimento (CDTC)~\footnote{\url{http://www.cdtc.org.br/}}. 
 
As pesquisas para levantamento de funcionalidades junto aos estudantes
da universidade foi realizada através de questionários com questões objetivas,
com possibilidade de justificativa das respostas.
%
Para a fase de desenvolvimento do trabalho, utilizamos algumas práticas das
metodologias ágeis como a realização de ciclos curtos de desenvolvimento,
programação em par, sempre que possível envolvendo membros da comunidade
mantenedora do Noosfero e colaboradores da UnB Gama, e com o conjunto de
testes automatizados (de unidade, funcional e de aceitação) para cada funcionalidade,
seguindo o desenvolvimento dirigido por comportamento (BDD)%
~\footnote{\url{http://en.wikipedia.org/wiki/Behavior-driven_development}},
quando possível.
 
Para apresentar nossa evolução e resultados, este texto, além desta introdução
este texto está organizado em capítulos. O Capítulo 2 apresenta uma visão geral
sobre mídias sociais. O Capítulo 3 apresenta
a plataforma para criação de redes sociais livros, Noosfero. O capítulo 4 a rede
Comunidade.UnB, as funcionalidades desenvolvidas por nós e o processo de
colaboração para o Noosfero.
