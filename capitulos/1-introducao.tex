\chapter{Introdução}

A utilização de redes sociais têm se tornado cada vez mais comum na vida das
pessoas, em especial nas gerações mais jovens. Neste contexto, não estamos nos
referindo apenas ao Facebook, Orkut e Twitter. 
%
Atualmente, é bem comum empresas terem suas próprias redes sociais para
promoverem a interação entre os seus funcionários e colaboradores. A ideia de
não depender de redes centralizadas, como é o caso das supracitadas, faz com que
as campanhas de Barack Obama~\footnote{\url{http://my.barackobama.com}}, nos
Estados Unidos, e DilmaRousseff~\footnote{\url{http://dilmanarede.com.br}}, no
Brasil, sejam exemplos da necessidade de autonomia das suas redes
(e informações delas) e como melhor explorar a Internet, que tem em sua
concepção ser descentralizada. 
%
Em especial, as tecnologias de mídia social exercem um papel importante para 
essa descentralização uma vez que possuem como característica a criação e o
compartilhamento de conteúdo de forma horizontal e colaborativa. 
%
Baseado nessa visão, a proposta desse trabalho é disponibilizar um ambiente
virtual que possibilite que os alunos da Universidade de Brasília criem e
compartilhem conhecimento de forma colaborativa e horizontal, sendo assim,
complementar ao modelo vertical de ensino adotado em sala de aula.

Estudos mostram que a grande maioria dos alunos de graduação faz uso de algum
tipo de rede social e que aqueles que participam de forma mais ativa delas
tendem a obter maior riqueza em suas relações sociais . 
%
Também, deve-se enfatizar que, está no cerne da criação das instituições de
ensino superior universalizar o conhecimento. Compartilhar e dar subsídios para
que o conhecimento seja disseminado e reproduzível é um dos pilares da ciência.
%
Adaptando essas ideias para o ponto de vista tecnológico, sites de rede social
são apenas uma camada das tecnologias de mídia social. A definição mais ampla de
tecnologias de mídia social inclui: a totalidade de "produtos" e "serviços"
digitais disponibilizados online; o comportamento social gerado pelo usuário; e
a permutação de conteúdo gerado primariamente pelos próprios usuários.

%------------------------------------------------------------------------------%

As tecnologias de mídia social também afetaram as universidades, de forma que o
número de perfis ou páginas oficiais de universidades é cada vez mais comum,
chegando a atingir 100\% das instituições de graduação americanas. As
universidades passaram a utilizar essas tecnologias como forma de promover uma
interação maior com seus integrantes. As tecnologias de mídia social deram às
instituição de ensino a oportunidade de divulgar as conquistas de seus alunos,
criando assim um sentimento de lealdade e ao mesmo tempo atraindo alunos em
potencial \cite{solis2008}. 
%
Além disso, foi constatado que alunos que utilizam tecnologias de mídia social
com o propósito de realizar atividades acadêmicas possuem maior nível de
engajamento nelas  \cite{ccsse2009}. No entanto, a maioria das redes sociais
utilizadas por esses alunos e universidades são as mesmas que tendem a querer
centralizar as atividades nelas e , em geral não 	costumam fornecer mecanismos
específicos para o intercâmbio de conhecimento científico e acadêmico.

%------------------------------------------------------------------------------%

Sob essa perspectiva das redes sociais no âmbito acadêmico, nós pretendemos
viabilizar a implantação de uma rede social própria para a Universidade de
Brasília baseada na ferramenta de software-livre 
Noosfero~\footnote{\url{http://noosfero.org/}}
na qual os usuários poderão publicar e compartilhar conteúdo livremente e
colaborar com a difusão dessa nuvem de conhecimento e ideias que a Universidade
nos proporciona.
%
Pretendemos também colaborar com a comunidade do Noosfero na implementação de um
protocolo de federação de forma a possibilitar no futuro a criação de uma rede
de colaboração federada na qual cada universidade poderá ter seu próprio nó, de
forma independente, mas ao mesmo tempo integrada aos demais.

O conceito de rede social federada é usado para indicar uma rede social autônoma
controlada por uma entidade mas que possibilita a interação, através de regras
acordadas, com usuários ou entidades de outras redes federadas sem a necessidade
de criar uma conta na segunda.

%------------------------------------------------------------------------------%

\section{Objetivos}

Esta seção apresenta os objetivos gerais e específicos deste TCC.

\subsection{Objetivos Gerais}

Neste trabalho de conclusão de curso, temos o objetivo de implementar as
principais funcionalidades para que uma rede de colaboração, como o Stoa da USP,
possa ser também disponibilizada na Universidade de Brasília. Além disso, queremos
aproveitar nossa "posição geográfica", com o fato do Noosfero ser uma plataforma
de software livre que tem seus principais desenvolvedores no Brasil, para
colaborarmos com o desenvolvimento dessa plataforma, interagindo diretamente com
essa comunidade. Planejamos contribuir com a implementação e adaptação de
um protocolo de federação para Noosfero, o que permitirá testarmos algumas
funcionalidades da federação entre as redes acadêmicas da USP e da versão que
iremos propomos como rede de colaboração da UnB.

%------------------------------------------------------------------------------%

\subsection{Específicos}

Os objetivos específicos desse trabalho são:

\begin{enumerate}
	
\item Identificar as tecnologias utilizadas pela UnB através das quais será
possível autenticar-se na comunidade.
\item Implantar uma instância do Noosfero e disponibilizar para a comunidade.
\item Integrar a instância do Noosfero às tecnologias cabíveis.
\item Adequar a instância do Noosfero ao padrão visual da UnB.
\item Levantar junto aos estudantes um conjunto de \textit{features} a serem
encorporadas ao Noosfero para disponibilizar um ambiente virtual adequado ao
ensino.
\item Implementar um conjunto de \textit{features} cabíveis.
\item Pesquisar a respeito de protocolos de federação.
\item Colaborar com a comunidade do Noosfero para a implementação do protocolo
de federação escolhido.

\end{enumerate}

%------------------------------------------------------------------------------%

\section{Organização do Trabalho}

Durante a primeira fase do projeto, o trabalho foi conduzido na forma de
pesquisas de fundamentação teórica sobre o uso de mídias sociais na educação
e sobre as tecnologias a serem utilizadas em nas fases mais avançadas.
Posteriormente, nos reunimos para instalar uma instância do Noosfero em um
servidor de testes disponibilizadas para nós pelo Centro de Difusão de
Tecnologia e Conhecimento (CDTC). 

As pesquisas para levantamento de \textit{features} junto aos estudantes
da universidade será feita através de questionários com questões objetivas
e subjetivas.
%
Para a fase de desenvolvimento do trabalho, utilizaremos algumas práticas das
metodologias ágeis como a realização de ciclos curtos de desenvolvimento,
programação em par, sempre que possível envolvendo membros da comunidade
mantenedora do Noosfero, e a realização de testes já nas primeiras fases do
desenvolvimento.

Para a primeira fase deste Trabalho de Conclusão de Curso, além desta introdução
este texto está organizado em capítulos. O Capítulo 2 apresenta a fundamentação
teórica sobre o uso de mídias sociais na educação. O Capítulo Seção 3 apresenta
a plataforma para criação de redes sociais livros, Noosfero, e aborda a
importância da adoção de software livre na educação. Por fim, o Capítulo 4
apresenta o estado atual da rede de colaboração da UnB, atualmente chamada
Comunidade UnB, e as atividades planejadas para a continuidade do projeto.
