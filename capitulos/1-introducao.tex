\chapter{Introdução}

A utilização de redes sociais têm se tornado cada vez mais comum na vida das pessoas, em especial nas gerações mais jovens. Neste contexto, não estamos nos referindo apenas ao Facebook, Orkut e Twitter. Atualmente, é bem comum empresas terem suas próprias redes sociais para promoverem a interação entre os seus funcionários e colaboradores. A ideia de não depender de redes centralizadas, como é o caso das supracitadas, faz com que as campanhas de Barack Obama~\footnote{\url{http://my.barackobama.com}}, nos Estados Unidos, e Dilma Rousseff~\footnote{\url{http://dilmanarede.com.br}}, no Brasil, sejam exemplos da necessidade de autonomia das suas redes (e informações delas) e como melhor explorar a Internet, que tem em sua concepção ser descentralizada. As tecnologias de mídia social são parte importante da internet e possuem como característica a criação e o compartilhamento de conteúdo de forma horizontal e colaborativa. A proposta desse trabalho é disponibilizar um ambiente virtual que possibilite que os alunos da Universidade de Brasília criem e compartilhem conhecimento de forma colaborativa e horizontal, sendo assim, complementar ao modelo vertical de ensino adotado em sala de aula.

Estudos mostram que a grande maioria dos alunos de graduação faz uso de algum tipo de rede social e que aqueles que participam de forma mais ativa delas tendem a obter maior riqueza em suas relações sociais . Também, deve-se enfatizar que, está no cerne da criação das instituições de ensino superior universalizar o conhecimento. Compartilhar e dar subsídios para que o conhecimento seja disseminado e reproduzível é um dos pilares da ciência. Adaptando essas ideias para o ponto de vista tecnológico, sites de rede social são apenas uma camada das tecnologias de mídia social. A definição mais ampla de tecnologias de mídia social inclui: a totalidade de "produtos" e "serviços" digitais disponibilizados online; o comportamento social gerado pelo usuário; e a permutação de conteúdo gerado primariamente pelos próprios usuários.
As tecnologias de mídia social também afetaram as universidades, de forma que o número de perfis ou páginas oficiais de universidades é cada vez mais comum, chegando a atingir 100\% das instituições de graduação americanas. As universidades passaram a utilizar essas tecnologias como forma de promover uma interação maior com seus integrantes. As tecnologias de mídia social deram às instituição de ensino a oportunidade de divulgar as conquistas de seus alunos, criando assim um sentimento de lealdade e ao mesmo tempo atraindo alunos em potencial \cite{solis2008}. Além disso, foi constatado que alunos que utilizam tecnologias de mídia social com o propósito de realizar atividades acadêmicas possuem maior nível de engajamento nelas  \cite{ccsse2009}. No entanto, a maioria das redes sociais utilizadas por esses alunos e universidades são as mesmas que tendem a querer centralizar as atividades nelas e , em geral, a interação social entre seus usuários não costumam fornecer mecanismos específicos para o intercâmbio de conhecimento científico e acadêmico.

Sob essa perspectiva das redes sociais no âmbito acadêmico, nós pretendemos viabilizar a implantação de uma rede social própria para a Universidade de Brasília baseada na ferramenta de software-livre Noosfero~\footnote{\url{http://noosfero.org/}} na qual os usuários poderão publicar e compartilhar conteúdo livremente e colaborar com a difusão dessa nuvem de conhecimento e ideias que a Universidade nos proporciona. Pretendemos também colaborar com a comunidade do Noosfero na implementação de um protocolo de federação de forma a possibilitar no futuro a criação de uma rede de colaboração federada na qual cada universidade poderá ter seu próprio nó, de forma independente, mas ao mesmo tempo integrada aos demais.

O conceito de rede social federada é usado para indicar uma rede social autônoma controlada por uma entidade mas que possibilita a interação, através de regras acordadas, com usuários ou entidades de outras redes federadas sem a necessidade de criar uma conta na segunda.

\section{Objetivos}

Gerais: Disponibilizar uma instância do Noosfero para a UnB que seja integrada com outras ferramentas da Universidade, como o Moodle, para permitir a colaboração entre seus integrantes. Colaborar com a comunidade desenvolvedora do Noosfero para implementar um protocolo de federação.

Específicos:

\begin{enumerate}
	
\item Identificar as tecnologias utilizadas pela UnB através das quais será possível autenticar-se na comunidade.
\item Implantar uma instância do Noosfero e disponibilizar para a comunidade.
\item Integrar a instância do Noosfero às tecnologias cabíveis.
\item Adequar a instância do Noosfero ao padrão visual da UnB.
\item Levantar junto aos estudantes um conjunto de \textit{features} a serem encorporadas ao Noosfero para disponibilizar um ambiente virtual adequado ao ensino.
\item Implementar um conjunto de \textit{features} cabíveis.
\item Pesquisar a respeito de protocolos de federação.
\item Colaborar com a comunidade do Noosfero para a implementação do protocolo de federação escolhido.


\end{enumerate}

%begin{itemize}
%end{itemize}
