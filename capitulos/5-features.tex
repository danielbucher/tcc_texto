\section{Funcionalidades}
%TODO explicar o formato de user stories e cenários de uso e correlacioanr
%com a sessão que trata sobre desenvolvimento de software livre
%conceituar histórias de usuário, BDD, cenário de uso e teste de aceitação
%enfatizar que os requisitos não serão apresentados da maneira tradicional
%(mais próximo de como são definidos nas equipes ágeis e nas comunidades de
%software livre.
Nesta seção descrevemos as funcionalidades desenvolvidas ao longo do segundo
semestre de 2013 que contaram com nossa contribuição e que contribuíram para
o crescimento da Comunidade UnB.


%------------------------------funcionalidade---------------------------------%
\subsection{Plugin Comunidade UnB}

\subsubsection*{Histórias de usuário}

O \textit{plugin} Comunidade UnB foi criado para suprir algumas necessidade de
integração de serviços fornecidos pela universidade e a rede de colaboração que
estamos propondo, como a autenticação via base de dados mantida pela universidade.
Assim o usuário vai poder ter acesso ao portal de comunidades através dos mesmos
dados utilizados para acessar outros serviços como por exemplo o serviço de
matrícula para alunos ou o serviço de lançamento de notas para professores.

Ao ativar o \textit{plugin}, os campos utilizadas para realizar o \textit{login}
no sistema serão alterados para os mesmo campos utilizados em sistemas da
universidade, matrícula para alunos e prefixo do correio eletrônico para
professores e funcionários técnico-administrativos, e a senha será a mesma.

%TODO adicionar LDAP e CPD nas siglas
Portanto o \textit{plugin} adiciona o campo matrícula e e-mail institucional
(caso o usuário deseje ter manter o campo de e-mail original para seu e-mail
pessoal) além de fazer a integração com o serviço de LDAP no qual o CPD mantém
os dados dos usuários. Contudo o usuário, durante seu primeiro \textit{login},
escolhe os campos nome de usuário, nome completo e e-mail.

Foi necessário também retirar a funcionalidade de alteração de senha uma vez
que queremos manter a compatibilidade entre o portal de comunidades proposto
e os demais serviços. Desta forma, caso o usuário deseje trocar sua senha 
mesmo deve procurar o CPD e solicitar a alteração.

Até o momento da escrita deste texto, conseguimos autorização para utilizar
apenas a base de dados que contém os dados dos alunos de forma que a
história de usuário e os cenários a seguir levarão em consideração apenas
a utilização do portal de comunidades pelos alunos, mas a funcionalidade
se mantém a mesma para professores e servidores técnico-administrativos que
queiram utilizar o serviço da Comunidade UnB.

%histórias
\begin{enumerate}

%--------------------------------história-------------------------------------%
\item Autenticação via LDAP da UnB

\textbf{Como} um aluno da Universidade Brasília

\textbf{Eu quero} me autenticar na rede através da minha matricula e senha
utilizada em outros sistemas.

\subsubsection*{Cenários de uso:}

%cenários
\begin{enumerate}

%---------------------------------cenário-------------------------------------%
\item Primeiro acesso

\textbf{Dado} que sou aluno da UnB

\textbf{E} possuo cadastro ativo na base de dados da UnB

\textbf{E} nunca utilizei o serviço da Comunidade UnB

\textbf{Quando} eu acessar o portal

\textbf{E} preencher os campos matricula e senha com a matrícula e a senha
fornecidas a mim pela universidade

\textbf{Então} deverei ser direcionado para uma página com o título
"Primeiro Acesso"

\textbf{E} deverei ver os campos "nome de usuário", "nome completo" e
"e-mail pessoal" em vazio.

%---------------------------------cenário-------------------------------------%
\item Registro

\textbf{Dado} que sou aluno da UnB

\textbf{E} me encontro na página de primeiro acesso da Comunidade UnB

\textbf{Quando} eu preencher os campos "nome de usuário",
"nome completo" e "e-mail pessoal" com "daniel.bucher", "Daniel Costa Bucher" e
"daniel.bucher88@gmail.com"

\textbf{E} clicar no botão "Registrar"

\textbf{Então} eu devo ser direcionado para meu perfil

\textbf{E} devo ver a url "<domínio>/daniel.bucher"

\textbf{E} devo ver "Daniel Costa Bucher" abaixo da imagem padrão de perfis
do Noosfero.

%---------------------------------cenário-------------------------------------%

\item Acesso

\textbf{Dado} que sou aluno da UnB

\textbf{E} já utilizei o serviço da Comunidade UnB

\textbf{Quando} eu entrar com minha matricula e senha fornecida pela universidade
nos campos adequados para autenticação

\textbf{E} clicar em entrar

\textbf{Então} eu devo me encontrar \textit{logado} no sistema com a minha conta.

%cenários
\end{enumerate}

%histórias
\end{enumerate}

\subsubsection*{Desenvolvedores responsáveis:}

%desenvolvedores

%TODO adicionar desenvolvedores responsáveis

%desenvolvedores

%------------------------------funcionalidade---------------------------------%
\subsection{Melhorias no \textit{plugin} de sub-organizações}

As histórias de usuários listadas nesta sub-seção dizem respeito a melhorias
no \textit{plugin} de sub-organizações do Noosfero e foram desenvolvidas em
conjunto com a Colivre e a equipe de estagiários do Portal da Faculdade do
Gama - FGA.

No Noosfero, uma organização é a abstração de uma entidade que pode assumir
o papel tanto de comunidade quanto de empreendimento. Os cenários de uso das
histórias a seguir foram especificados utilizando comunidades como exemplo,
mas as alterações realizadas afetam da mesma forma a relação entre usuários
e empreendimentos.

%TODO colocar figura da relação de heranças organização -> comunidade/empreendimento

O termo ``organização mãe'' é utilizado para designar uma organização que
possua sub-organizações, também chamadas de organizações filhas. Conforme
descrito acima, uma organização mãe pode ser tanto uma comunidade quanto um
empreendimento.

\subsubsection*{Histórias de usuário:}

%histórias de usuário
\begin{enumerate}

%--------------------------------história-------------------------------------%
\item Listar organizações `mãe' na lista de organizações de um usuário

Esta história de usuário está mapeada no ActionItem de número 2825\footnote{\url
{https://noosfero.org/Development/ActionItem2825}}.

\textbf{Como} um usuário

\textbf{Eu quero} visualizar organizações mãe de sub-organizações que faço
parte junto à lista de minhas organizações.

\subsubsection*{Cenários de uso:}

%cenários
\begin{enumerate}

%---------------------------------cenário-------------------------------------%
\item Ver comunidade `mãe' na página `Gerenciar meus grupos'

\textbf{Dado} que eu estou logado com a usuário `ze'

\textbf{E} `ze' não é membro da comunidade UnB

\textbf{E} `ze' é membro da sub-comunidade de Unb, FGA

\textbf{Quando} eu navegar até a página `Gerenciar meus grupos'
(/myprofile/ze/memberships)

\textbf{Então} eu tenho que ver a comunidade UnB listada junto às
demais comunidades que faço parte.

%---------------------------------cenário-------------------------------------%
\item Ver comunidade `mãe' na página `Comunidades de ze'

\textbf{Dado} que eu estou logado com a usuário `ze'

\textbf{E} `ze' não é membro da comunidade UnB

\textbf{E} `ze' é membro da sub-comunidade de Unb, FGA

\textbf{Quando} eu navegar até a página `Comunidades de ze'
(/profile/ze/communities)

\textbf{Então} eu tenho que ver a comunidade UnB listada junto às
demais comunidades que faço parte.

%cenários
\end{enumerate}

%--------------------------------história-------------------------------------%
\item Bloco de organizações relacionadas

Esta história de usuário está mapeada no ActionItem de número 2499\footnote{\url
{https://noosfero.org/Development/ActionItem2499}}.

\textbf{Para} ter acesso às organizações relacionadas a uma organização

\textbf{Como} um usuário

\textbf{Eu quero} visualizar um bloco que liste todas as organizações
relacionadas à organização atual.

\subsubsection*{Cenários de uso:}

%cenários
\begin{enumerate}

%---------------------------------cenário-------------------------------------%
\item Adicionar um bloco de sub-organizações

\textbf{Dado} que eu estou logado com meu usuário

\textbf{E} meu usuário é administrador da comunidade X

\textbf{Quando} eu navegar até o painel de controle da comunidade X

\textbf{E} eu clicar em "Editar blocos laterais"

\textbf{E} eu clicar em "Adicionar bloco"

\textbf{Então} eu tenho que ver a opção 'Organizações Relacionadas'.

%---------------------------------cenário-------------------------------------%
\item Listar todas as sub-organizações na organização `mãe'

\textbf{Dado} que eu estou logado com meu usuário

\textbf{E} meu usuário é membro da comunidade X

\textbf{E} e a comunidade Y é uma sub-comunidade de X

\textbf{E} e o empreendimento Z é um sub-empreendimento de X

\textbf{E} a comunidade X possua um bloco de organizações relacionadas

\textbf{Quando} eu navegar até a página da comunidade X

\textbf{Então} eu tenho que ver um bloco com o título 'Sub organizações'.

\textbf{E} eu tenho que ver um \textit{link} para a sub-comunidade Y neste
bloco

\textbf{E} eu tenho que ver um \textit{link} para o sub-empreendimento Z neste
bloco.

%---------------------------------cenário-------------------------------------%
\item Listar apenas as sub-comunidades organização `mãe'

\textbf{Dado} que eu estou logado com meu usuário

\textbf{E} meu usuário é membro da comunidade X

\textbf{E} e a comunidade Y é uma sub-comunidade de X

\textbf{E} e o empreendimento Z é um sub-empreendimento de X

\textbf{E} a comunidade X possua um bloco de organizações relacionadas

\textbf{E} o bloco esteja configurado para mostrar apenas comunidades

\textbf{Quando} eu navegar até a página da comunidade X

\textbf{Então} eu tenho que ver um bloco com o título 'Sub comunidades'.

\textbf{E} eu tenho que ver um \textit{link} para a sub-comunidade Y neste
bloco

\textbf{E} eu não tenho que ver um \textit{link} para o sub-empreendimento Z
neste bloco.

%---------------------------------cenário-------------------------------------%
\item Visualizar página de sub-organizações de uma organização `mãe'

\textbf{Dado} que eu estou logado com meu usuário

\textbf{E} meu usuário é membro da comunidade X

\textbf{E} e a comunidade Y é uma sub-comunidade de X

\textbf{E} e o empreendimento Z é um sub-empreendimento de X

\textbf{E} a comunidade X possua um bloco de organizações relacionadas

\textbf{Quando} eu navegar até a página da comunidade X

\textbf{E} eu clicar no link "Ver todos" do bloco de organizações relacionadas

\textbf{Então} eu tenho que ver uma página de sub-organizações

\textbf{E} eu tenho que ver um \textit{link} para a sub-comunidade Y neste
bloco

\textbf{E} eu tenho que ver um \textit{link} para o sub-empreendimento Z
neste bloco.

%---------------------------------cenário-------------------------------------%
\item Visualizar página de sub-comunidades de uma organização `mãe'

\textbf{Dado} que eu estou logado com meu usuário

\textbf{E} meu usuário é membro da comunidade X

\textbf{E} e a comunidade Y é uma sub-comunidade de X

\textbf{E} e o empreendimento Z é um sub-empreendimento de X

\textbf{E} a comunidade X possua um bloco de organizações relacionadas

\textbf{E} o bloco esteja configurado para mostrar apenas comunidades

\textbf{Quando} eu navegar até a página da comunidade X

\textbf{E} eu clicar no link "Ver todos" do bloco de organizações relacionadas

\textbf{Então} eu tenho que ver uma página de sub-organizações

\textbf{E} eu tenho que ver um \textit{link} para a sub-comunidade Y neste
bloco

\textbf{E} eu não tenho que ver um \textit{link} para o sub-empreendimento Z
neste bloco.

%---------------------------------cenário-------------------------------------%
\item Listar todas as organizações `mãe' de uma organização

\textbf{Dado} que eu estou logado com meu usuário

\textbf{E} meu usuário é membro da comunidade X

\textbf{E} e a comunidade Y é `mãe' de X

\textbf{E} e o empreendimento é `pai' de X

\textbf{E} a comunidade X possua um bloco de organizações relacionadas

\textbf{Quando} eu navegar até a página da comunidade X

\textbf{Então} eu tenho que ver um bloco com o título 'Organizações pais'.

\textbf{E} eu tenho que ver um \textit{link} para a sub-comunidade Y neste
bloco

\textbf{E} eu tenho que ver um \textit{link} para o sub-empreendimento Z neste
bloco.

%cenários
\end{enumerate}

%--------------------------------história-------------------------------------%
\item Visualização completa nas páginas de organização relacionadas

\textbf{Como} um usuário

\textbf{Eu quero} visualizar informações sobre as organizações relacionadas nas
páginas de organizações relacionadas

\textbf{Para} me informar sobre elas sem precisar visitar a página de cada uma.

\subsubsection*{Cenários de uso:}

%cenários
\begin{enumerate}

%---------------------------------cenário-------------------------------------%
\item Visualizar modo completo na página de sub-organizações

\textbf{Dado} a comunidade X

\textbf{E} a comunidade Y é `filha' de X

\textbf{E} o empreendimento Z é `filho' de X

\textbf{Quando} eu navegar até a página de sub-organizações de X

\textbf{E} eu clicar na opção de visualização completa

\textbf{Então} eu tenho que ver as informações

%cenários
\end{enumerate}

%histórias de usuário
\end{enumerate}

%-----------------------------------------------------------------------------%

\subsubsection*{Desenvolvedores responsáveis:}

Os seguintes colaboradores do Noosfero participaram do desenvolvimento desta
funcionalidade:

%desenvolvedores
\begin{enumerate}

\item Aurélio A. Heckert - Colivre

\item Daniel Bucher - UnB

\item Equipe do Portal FGA - UnB

%desenvolvedores
\end{enumerate}

%------------------------------funcionalidade---------------------------------%
\subsection{\textit{Plugin} de bloco de video}

Esta funcionalidade foi desenvolvido pela equipe de estagiários do Portal
da FGA e está mapeada no ActionItem de número 2823\footnote{\url{https://
noosfero.org/Development/ActionItem2823}} e consiste em um \textit{plugin}
que adicionar um novo tipo de bloco no Noosfero, o VideoBlock.

Este bloco incorpora vídeos de plataformas externas, atualmente o Vimeo
e o Youtube, dentro de seu conteúdo.

%histórias
\begin{enumerate}

%--------------------------------história-------------------------------------%
\item Bloco de video

\textbf{Para} adicionar vídeos em um perfil do Noosfero

\textbf{Como} um usuário

\textbf{Eu quero} ter um bloco em que eu possa adicionar um vídeo de plataformas
como o Youtube\footnote{\url{https://youtube.com}} e o Vimeo\footnote{\url{https://
vimeo.com}}.

\subsubsection*{Cenários de uso:}

%cenários
\begin{enumerate}

%---------------------------------cenário-------------------------------------%
\item Adicionar bloco

\textbf{Dado} que estou logado como o usuário `Zé'

\textbf{E} o \textit{plugin} Video esteja atualizado

\textbf{Quando} eu navegar até a página "Editar blocos laterais" do meu perfil

\textbf{E} clicar em "Adicionar bloco"

\textbf{E} selecionar "Bloco de vídeo"

\textbf{E} clicar em "Adicionar"

\textbf{Então} eu devo ver um bloco de vídeo sem conteúdo na área principal.


%---------------------------------cenário-------------------------------------%
\item Adicionar vídeo do Youtube

\textbf{Dado} que estou logado como o usuário `Zé'

\textbf{E} possui um bloco de vídeo no meu perfil

\textbf{Quando} eu navegar até a página de "Editar blocos laterais" do meu perfil

\textbf{E} clicar em "Editar" no bloco de vídeo

\textbf{E} adicionar a URL de um video do Youtube

\textbf{E} preencher os campos "Largura" e "Altura" com 536 e 360
respectivamente

\textbf{E} clicar em "Salvar"

\textbf{Então} eu devo ver o vídeo incorporado na área principal do meu
perfil na resolução "536x360".


%---------------------------------cenário-------------------------------------%

\item Adicionar vídeo do Vimeo

\textbf{Dado} que estou logado como o usuário `Zé'

\textbf{E} possui um bloco de vídeo no meu perfil

\textbf{Quando} eu navegar até a página de "Editar blocos laterais" do meu perfil

\textbf{E} clicar em "Editar" no bloco de vídeo

\textbf{E} adicionar a URL de um video do Vimeo

\textbf{E} preencher os campos "Largura" e "Altura" com 536 e 360
respectivamente

\textbf{E} clicar em "Salvar"

\textbf{Então} eu devo ver o vídeo incorporado na área principal do meu
perfil na resolução "536x360".

%cenários
\end{enumerate}

%histórias
\end{enumerate}

\subsubsection*{Desenvolvedores responsáveis}

Os seguintes colaboradores do Noosfero participaram do desenvolvimento desta
funcionalidade:

%desenvolvedores
\begin{enumerate}

\item Daniel Bucher - UnB

\item Equipe do Portal FGA - UnB

%desenvolvedores
\end{enumerate}
 
%------------------------------funcionalidade---------------------------------%
\subsection{Usabilidade do bloco ``\textit{links}''}

%TODO Dividir em mais de uma user story

Implementar melhorias no Bloco ``Links'', para que fique muito mais intuitivo para um usuário personalizar o seu menu lateral. Incluindo;
- Mover itens por meio de ``Drag and Drop''
- Interface fácil de incluir links para conteúdo próprio

Implementar melhorias no bloco ``links'' para aumentar a usabilidade:
  - Permitir reorganização de link via ``drag and drop''
  - Interface fácil de inserir links para conteúdo

\subsubsection*{Histórias de usuário}

\begin{enumerate}

%--------------------------------história-------------------------------------%
\item Mover itens por \textit{Drag and Drop}

\textbf{Como} um usuário do Noosfero

\textbf{Eu quero} reordenar os items de um bloco de \textit{links} por meio de
\textit{Drag and Drop}.

\subsubsection*{Cenários de uso}

\begin{enumerate}

\item Arrastar um item

\textbf{Dado} que eu estou logado com meu usuário

\textbf{E} estou na página de edição de blocos laterais

\textbf{E} eu clico no botão de edição do bloco padrão de links

\textbf{E} eu veja 'Perfil' em cima de 'Galeria de imagens'

\textbf{Quando} arrastar 'Galeria de imagens' para cima de 'Perfil'

\textbf{E} clicar no botão 'Salvar'

\textbf{E} eu for para a página do meu perfil

\textbf{Então} eu tenho que ver 'Galeria de imagens' em cima de 'Perfil'
no bloco de links padrão do perfil.

\end{enumerate}

%--------------------------------história-------------------------------------%
\item \textit{Auto-complete} ao adicionar novos itens
%TODO elaborar user story

\end{enumerate}

%------------------------------funcionalidade---------------------------------%
\subsection{Convite para participação de comunidades}

Esta funcionalidade prevê melhorias no sistema de convite para participar de
uma comunidade. Antes de sua implementação, os convites eram realizados
exclusivamente através de e-mail, seja entrando com o e-mail da pessoa
diretamente ou pesquisando na lista de contatos de um e-mail passado. Planejamos
esta funcionalidade para incluir uma interface através da qual um usuário
possa adicionar facilmente outros usuários da sua lista de amigos pesquisando
pelo nome, ou, caso deseje convidar um usuário do qual não seja amigo, ou alguém
que ainda não tenha cadastro na rede através do e-mail.

\subsubsection*{Histórias de usuário}

\begin{enumerate}

%--------------------------------história-------------------------------------%
\item Convidar usuários da rede

\textbf{Para} facilitar a criação de convites para juntar-se a comunidade

\textbf{Como} um membro de comunidade com permissão para convidar membros

\textbf{Eu quero} convidar usuários através de uma interface que busque na lista
de perfis de pessoas do ambiente e auto-complete conforme for digitando.

\subsubsection*{Cenários de uso}

\begin{enumerate}

\item Procurar um usuário cadastrado

\textbf{Dado} que eu estou logado com meu usuário

\textbf{E} meu usuário é administrador da comunidade UnB

\textbf{E} o usuário `Zé' existe

\textbf{E} o usuário `Zé' não é membro da comunidade UnB

\textbf{E} eu estou na página de convidar amigos para a comunidade UnB

\textbf{Quando} eu digitar `Zé' no campo de convidar novos amigos

\textbf{Então} eu devo ver um \textit{token}\footnote{Neste contexto, um
\textit{token} representa um objeto, no caso um perfil de usuário, que possua
um nome que bata com o campo digitado.} representando o usuário `Zé'.

\end{enumerate}

\end{enumerate}