\begin{resumo}[Abstract]
  \begin{otherlanguage*}{english}  

  This course conclusion work presents the results of a study to enable the
  implantation of a collaborative network for the University of
  Brasilia (UNB), which acts as a virtual environment for the creation and
  sharing of knowledge in a collaborative and horizontal way.
  %
  For this, we choose to use the Brazilian free social networks platform
  Noosfero, understanding that it satisfies the immediate needs of the
  project, according to studies done by University of São Paulo, when it
  adopted the same.
  %  
  Besides the implementation itself at UNB, this study includes an requirements
  elicitation and the implementation of a set of features and enhancements to
  the platform in question, so that would meet the basic needs for us to perform
  case studies with UnB Gama students.
  %
  Thus, indicating how we can formalize the Comunidade.UnB network and
  as well as to what are the next steps that best meets the public at this
  university.
  %
  Additionally, efforts and knowledge acquired in this work were transferred
  to a team of developers at UNB Gama, which will provide continuity to the
  deployment and implementation of this network in UNB, in 2014.
   

  \vspace{\onelineskip}
 
  \noindent 
  \textbf{Key-words}: social networking. open-source software. functional requirements. agile methods. distributed software development.
  \end{otherlanguage*}
\end{resumo}
