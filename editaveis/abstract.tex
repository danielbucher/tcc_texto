\begin{resumo}[Abstract]
 \begin{otherlanguage*}{english}   
   This undergraduate work aims to present a theoric justification and to raise
   the requirements needed to enable the implantation of a colaboration network
   for the University of Brasília that acts as a virtual environment that allows
   users to create and share knowledge in a collaborative and horizontal way.
   %
   This approach allows the rupture of the professor-student relationship and
   causes the student to act more as an author or co-author for that
   knowledge.
   %
   In this context, we selected the brazilian free social networking tool
   Noosfero considered that it meets the project's immediate needs. Furthermore,
   we want to use our favorable geographical position to contribute to the
   community maintining the plaform since it has most of its developers located
   in Brazil.
   %
   We also intend to collaborate in the implementation of a federation protocol
   for Noosfero that enables communication and exchange of content between
   other instances of this or other networks that implement the same protocol,
   thus becoming a node within a potential collaborative network formed by
   various educational institutions, each with it's own network.
   

   \vspace{\onelineskip}
 
   \noindent 
   \textbf{Key-words}: social networking. open-source. federation.
 \end{otherlanguage*}
\end{resumo}